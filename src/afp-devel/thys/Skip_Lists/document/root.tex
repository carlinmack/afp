\documentclass[11pt,a4paper]{article}
\usepackage{isabelle,isabellesym}
\usepackage{amsfonts, amsmath, amssymb}

% this should be the last package used
\usepackage{pdfsetup}

% urls in roman style, theory text in math-similar italics
\urlstyle{rm}
\isabellestyle{it}


\begin{document}

\title{Randomised Skip Lists}
\author{Max W. Haslbeck, Manuel Eberl}
\maketitle

\begin{abstract}
Skip lists are sorted linked lists enhanced with shortcuts and are an alternative to binary search trees \cite{pugh1989skip}.
A skip lists consists of multiple levels of sorted linked lists where a list on level $n$ is a subsequence of the list on level $n - 1$.
In the ideal case, elements are \emph{skipped} in such a way that a lookup in a skip lists takes $\mathcal{O}(\log{n})$ time.
In a randomised skip list the skipped elements are choosen randomly.

This entry contains formalized proofs of the textbook results about the expected height and the expected length of a search path in a randomised skip list \cite{motwani1995}.
\end{abstract}

\tableofcontents
\newpage
\parindent 0pt\parskip 0.5ex

\input{session}

\bibliographystyle{abbrv}
\bibliography{root}

\end{document}

%%% Local Variables:
%%% mode: latex
%%% TeX-master: t
%%% End:
