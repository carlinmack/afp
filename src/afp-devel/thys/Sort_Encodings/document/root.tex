\documentclass[11pt,a4paper]{article}
\usepackage{isabelle,isabellesym}

% this should be the last package used
\usepackage{pdfsetup}

% urls in roman style, theory text in math-similar italics
\urlstyle{rm}
\isabellestyle{it}


\begin{document}

\title{Sound and Complete Sort Encodings \\for First-Order Logic}
\author{Jasmin Christian Blanchette and Andrei Popescu}
\date{}
\maketitle

\begin{abstract}
This is a formalization of the soundness and completeness properties for
various efficient encodings of sorts in unsorted first-order logic
used by Isabelle's Sledgehammer tool.

The results are reported in
\cite[\S2,3]{blanchette-et-al-2013-types-conf}, and the formalization itself
is presented in \cite[\S3--5]{froc}.
%
The encodings proceed as follows:\ a many-sorted problem is decorated with (as
few as possible) tags or guards that make the problem monotonic; then sorts can
be soundly erased.
%
The proofs rely on monotonicity criteria recently introduced by
Claessen, Lilliestr{\"o}m, and Smallbone \cite{claessen-et-al-2011}.

The development employs a formalization of many-sorted first-order logic in
clausal form (clauses, structures, and the basic properties of the satisfaction
relation), which could be of interest as the starting point for other
formalizations of first-order logic metatheory.

\end{abstract}

\bibliographystyle{abbrv}
\bibliography{root}

\newpage
\tableofcontents


% sane default for proof documents
\parindent 0pt\parskip 0.5ex



% generated text of all theories
\input{session}

% optional bibliography


\end{document}

%%% Local Variables:
%%% mode: latex
%%% TeX-master: t
%%% End:
