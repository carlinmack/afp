\documentclass[11pt,a4paper]{article}
\usepackage{isabelle,isabellesym}

% further packages required for unusual symbols (see also
% isabellesym.sty), use only when needed

%\usepackage{amssymb}
  %for \<leadsto>, \<box>, \<diamond>, \<sqsupset>, \<mho>, \<Join>,
  %\<lhd>, \<lesssim>, \<greatersim>, \<lessapprox>, \<greaterapprox>,
  %\<triangleq>, \<yen>, \<lozenge>

%\usepackage{eurosym}
  %for \<euro>

%\usepackage[only,bigsqcap]{stmaryrd}
  %for \<Sqinter>

%\usepackage{eufrak}
  %for \<AA> ... \<ZZ>, \<aa> ... \<zz> (also included in amssymb)

%\usepackage{textcomp}
  %for \<onequarter>, \<onehalf>, \<threequarters>, \<degree>, \<cent>,
  %\<currency>

% this should be the last package used
\usepackage{pdfsetup}

% urls in roman style, theory text in math-similar italics
\urlstyle{rm}
\isabellestyle{it}

% for uniform font size
%\renewcommand{\isastyle}{\isastyleminor}

\usepackage[english]{babel}
% for \frqq (whatever that actually is)

\begin{document}

\title{An implementation of ROBDDs for Isabelle/HOL}
\author{Julius Michaelis and Maximilian Haslbeck and Peter Lammich and Lars Hupel}
\maketitle

\begin{abstract}
	We present a verified and executable implementation of ROBDDs in Isabelle/HOL.
	Our implementation relates pointer-based computation in the Heap monad to operations on an abstract definition of boolean functions.
	Internally, we implemented the if-then-else combinator in a recursive fashion, following the Shannon decomposition of the argument functions.
	The implementation mixes and adapts known techniques and is built with efficiency in mind.
\end{abstract}

\tableofcontents

% sane default for proof documents
\parindent 0pt\parskip 0.5ex

\section{Preface}
This work is not the first to deal with BDDs in Isabelle/HOL.
Ortner and Schirmer have formalized BDDs in~\cite{BDD-AFP} and proved the correctness of an algorithm that transforms arbitrary BDDs to ROBDDs.
However, their specification does not provide efficiently executable algorithms on BDDs.
Giorgino and Strecker have presented efficiently executable algorithms for ROBDDs~\cite{giorgino2012correctness} by reducing their arguments to manipulating edges of graphs.
However, they have, to the best of our knowledge, not made their theory files available.
Thus, no library for efficient computation on (RO)BDDs in Isabelle/HOL existed.
Our work is a response to that situation.

The theoretic background of the implementation is mostly based on~\cite{brace1991efficient}.


% generated text of all theories
\input{session}

% optional bibliography
\bibliographystyle{abbrv}
\bibliography{root}

\end{document}

%%% Local Variables:
%%% mode: latex
%%% TeX-master: t
%%% End:
