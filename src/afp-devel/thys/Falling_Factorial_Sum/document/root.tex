\documentclass[11pt,a4paper]{article}
\usepackage{isabelle,isabellesym}

% further packages required for unusual symbols (see also
% isabellesym.sty), use only when needed

%\usepackage{amssymb}
  %for \<leadsto>, \<box>, \<diamond>, \<sqsupset>, \<mho>, \<Join>,
  %\<lhd>, \<lesssim>, \<greatersim>, \<lessapprox>, \<greaterapprox>,
  %\<triangleq>, \<yen>, \<lozenge>

%\usepackage{eurosym}
  %for \<euro>

%\usepackage[only,bigsqcap]{stmaryrd}
  %for \<Sqinter>

%\usepackage{eufrak}
  %for \<AA> ... \<ZZ>, \<aa> ... \<zz> (also included in amssymb)

%\usepackage{textcomp}
  %for \<onequarter>, \<onehalf>, \<threequarters>, \<degree>, \<cent>,
  %\<currency>

% this should be the last package used
\usepackage{pdfsetup}

% urls in roman style, theory text in math-similar italics
\urlstyle{rm}
\isabellestyle{it}

% for uniform font size
%\renewcommand{\isastyle}{\isastyleminor}

\begin{document}

\title{The Falling Factorial of a Sum}
\author{Lukas Bulwahn}
\maketitle

\begin{abstract}

This entry shows that the falling factorial of a sum can be computed with an expression
using binomial coefficients and the falling factorial of its summands. The entry provides
three different proofs: a combinatorial proof, an induction proof and an algebraic proof
using the Vandermonde identity.

The three formalizations try to follow their informal presentations from a Mathematics Stack
Exchange page~\cite{mse-1, mse-2, mse-3, mse-4} as close as possible. The induction and
algebraic formalization end up to be very close to their informal presentation, whereas the
combinatorial proof first requires the introduction of list interleavings, and significant
more detail than its informal presentation.

\end{abstract}

\tableofcontents

% sane default for proof documents
\parindent 0pt\parskip 0.5ex

% generated text of all theories
\input{session}

\section{Note on Copyright Licensing}

The initial material of the informal proof for this formalisation is provided
on Mathematics Stack Exchange under the Creative Commons Attribution-ShareAlike
3.0 Unported license (CC BY-SA 3.0; \url{https://creativecommons.org/licenses/by-sa/3.0/}),
which is pointed out on the the Mathematics Stack Exchange terms of use
at~\url{https://stackexchange.com/legal/terms-of-service}.

The two main proofs, the induction and the algebraic proof in this AFP entry
are (even textually) very close to the initial material from Mathematics Stack Exchange.

In case the two Isabelle proofs are judged to build upon the main proofs from
Mathematics Stack Exchange, the CC BY-SA 3.0 license requires that these proofs
must be available under the same license, and hence, these proofs are consequently
licensed under CC BY-SA 3.0. In case the two Isabelle proofs are not judged to build
upon the material from Mathematics Stack Exchange, I as an author provide them under
the 3-Clause BSD License~(\url{https://opensource.org/licenses/BSD-3-Clause}) to
allow their seemless integration into the Isabelle repository at any point in time.

All other content that does not build upon the material from Mathematics Stack Exchange
is licensed under the 3-clause BSD License, and can be copied, moved or integrated in
other work licensed under the 3-clause BSD License without further consideration of the
different obligations of the existing copyright licensing.

\nocite{*}

\bibliographystyle{abbrv}
\bibliography{root}

\end{document}

%%% Local Variables:
%%% mode: latex
%%% TeX-master: t
%%% End:
