\documentclass[11pt,a4paper]{article}
\usepackage[T1]{fontenc}
\usepackage{amssymb}
\usepackage{isabelle,isabellesym}
\usepackage[english]{babel}  % for guillemots

% this should be the last package used
\usepackage{pdfsetup}

% urls in roman style, theory text in math-similar italics
\urlstyle{rm}
\isabellestyle{it}

\begin{document}

\title{Roth's Theorem on Arithmetic Progressions}
\author{Chelsea Edmonds, Angeliki Koutsoukou-Argyraki and Lawrence C. Paulson\\
Computer Laboratory, University of Cambridge CB3 0FD\\
\texttt{\{cle47,ak2110,lp15\}@cam.ac.uk}}

\maketitle

\begin{abstract}
We formalise a proof of Roth's Theorem on Arithmetic Progressions, a major result in additive 
combinatorics on the existence of 3-term arithmetic progressions in subsets of natural numbers.
To this end, we follow a proof using graph regularity. We employ our recent formalisation of Szemer\'{e}di's 
Regularity Lemma, a major result in extremal graph theory, which we use here to prove
the Triangle Counting Lemma and the Triangle Removal Lemma. 
Our sources are Yufei Zhao's MIT lecture notes ``Graph Theory and Additive Combinatorics''%
\footnote{\url{https://ocw.mit.edu/courses/mathematics/18-217-graph-theory-and-additive-combinatorics-fall-2019/lecture-notes/MIT18_217F19_ch3.pdf} and \url{https://yufeizhao.com/gtac/gtac17.pdf}}
and W.T. Gowers's Cambridge lecture notes ``Topics in Combinatorics''.%
\footnote{\url{https://www.dpmms.cam.ac.uk/~par31/notes/tic.pdf}}
We also refer to the University of Georgia notes by Stephanie Bell and Will Grodzicki
``Using Szemerédi's Regularity Lemma to Prove Roth's Theorem''.%
\footnote{\url{http://citeseerx.ist.psu.edu/viewdoc/summary?doi=10.1.1.432.327}}
\end{abstract}

\tableofcontents

\subsection*{Acknowledgements}
The authors were supported by the ERC Advanced Grant ALEXANDRIA (Project 742178) funded by the European Research Council. 

\newpage

% include generated text of all theories
\input{session}

\end{document}
