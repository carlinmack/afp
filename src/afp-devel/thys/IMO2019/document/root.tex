\documentclass[11pt,a4paper]{article}
\usepackage{isabelle,isabellesym}
\usepackage{amsfonts, amsmath, amssymb}

% this should be the last package used
\usepackage{pdfsetup}

% urls in roman style, theory text in math-similar italics
\urlstyle{rm}
\isabellestyle{it}

\begin{document}

\title{International Mathematical Olympiad 2019}
\author{Manuel Eberl}
\maketitle

\begin{abstract}
This entry contains formalisations of the answers to three of the six problem of the International Mathematical Olympiad 2019, namely Q1, Q4, and Q5.
The reason why these problems were chosen is that they are particularly amenable to formalisation: they can be solved with minimal use of libraries. The remaining three concern geometry and graph theory, which, in the author's opinion, are more difficult to formalise resp.\ require a more complex library.
\end{abstract}

\tableofcontents
\newpage
\parindent 0pt\parskip 0.5ex

\input{session}

\nocite{imo2019}
\bibliographystyle{abbrv}
\bibliography{root}

\end{document}

%%% Local Variables:
%%% mode: latex
%%% TeX-master: t
%%% End:
