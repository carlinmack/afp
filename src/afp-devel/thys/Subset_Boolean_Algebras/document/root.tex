\documentclass[11pt,a4paper]{article}

\usepackage{isabelle,isabellesym}
\usepackage{amssymb,ragged2e}
\usepackage{pdfsetup}

\isabellestyle{it}
\renewenvironment{isamarkuptext}{\par\isastyletext\begin{isapar}\justifying\color{blue}}{\end{isapar}}
\renewcommand\labelitemi{$*$}

\begin{document}

\title{A Hierarchy of Algebras for Boolean Subsets}
\author{Walter Guttmann and Bernhard M\"oller}
\maketitle

\begin{abstract}
  We present a collection of axiom systems for the construction of Boolean subalgebras of larger overall algebras.
  The subalgebras are defined as the range of a complement-like operation on a semilattice.
  This technique has been used, for example, with the antidomain operation, dynamic negation and Stone algebras.
  We present a common ground for these constructions based on a new equational axiomatisation of Boolean algebras.
\end{abstract}

\tableofcontents

\section{Overview}

A Boolean algebra often arises as a subalgebra of some overall algebra.
To avoid introducing a separate type for the subalgebra, the overall algebra can be enriched with a special operation leading into the intended subalgebra and axioms to guarantee that the range of this operation has a Boolean structure.
Examples for this are the antidomain operation in idempotent (left) semirings \cite{DesharnaisStruth2008b,DesharnaisStruth2008a,DesharnaisStruth2011}, dynamic negation \cite{Hollenberg1997}, the operation yielding tests in \cite{Guttmann2012c,GuttmannStruthWeber2011b}, and the pseudocomplement operation in Stone algebras \cite{Frink1962,Graetzer1971,Guttmann2018c}.
The present development looks at a common ground pattern.

In Sections 2 and 3 we relate various axiomatisations of Boolean algebras from the literature and present a new equational one tailored to our needs.
Section 4 adapts this for the construction of Boolean subalgebras of larger overall algebras.
In Section 5 we add successively stronger assumptions to the overall algebra.
Sections 6, 7 and 8 show how Stone algebras, domain semirings and antidomain semirings fit into this hierarchy.

This Isabelle/HOL theory formally verifies results in \cite{GuttmannMoeller2020}.
See that paper for further details and related work.
Some proofs in this theory have been translated from proofs found by Prover9 \cite{McCune2010} using a program we wrote.

\begin{flushleft}
\input{session}
\end{flushleft}

\bibliographystyle{abbrv}
\bibliography{root}

\end{document}

