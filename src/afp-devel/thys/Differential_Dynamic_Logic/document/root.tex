\documentclass[11pt,a4paper]{article}
\usepackage[T1]{fontenc}
\usepackage{isabelle,isabellesym}

% further packages required for unusual symbols (see also
% isabellesym.sty), use only when needed

%\usepackage{amssymb}
  %for \<leadsto>, \<box>, \<diamond>, \<sqsupset>, \<mho>, \<Join>,
  %\<lhd>, \<lesssim>, \<greatersim>, \<lessapprox>, \<greaterapprox>,
  %\<triangleq>, \<yen>, \<lozenge>

%\usepackage{eurosym}
  %for \<euro>

%\usepackage[only,bigsqcap]{stmaryrd}
  %for \<Sqinter>

%\usepackage{eufrak}
  %for \<AA> ... \<ZZ>, \<aa> ... \<zz> (also included in amssymb)

%\usepackage{textcomp}
  %for \<onequarter>, \<onehalf>, \<threequarters>, \<degree>, \<cent>,
  %\<currency>

% this should be the last package used
\usepackage{pdfsetup}

% urls in roman style, theory text in math-similar italics
\urlstyle{rm}
\isabellestyle{it}

% for uniform font size
%\renewcommand{\isastyle}{\isastyleminor}


\begin{document}

\title{Differential-Dynamic-Logic}
\author{Brandon Bohrer}
\maketitle

\begin{abstract}
 We formalize differential dynamic logic, a logic for proving
 properties of hybrid systems. The proof calculus in this
 formalization is based on the uniform substitution principle. We show
 it is sound with respect to our denotational semantics, which
 provides increased confidence in the correctness of the KeYmaera X
 theorem prover based on this calculus. As an application, we include
 a proof term checker embedded in Isabelle/HOL with several example
 proofs.

Published in \cite{BohrerCPP17}
\end{abstract}

  We present a formalization of a uniform substitution calculus for
  differential dynamic logic (dL). In this calculus, the soundness of dL
  proofs is reduced to the soundness of a finite number of axioms, standard
  propositional rules and a central \textit{uniform substitution} rule for
  combining axioms. We present a formal definition for the denotational
  semantics of dL and prove the uniform substitution calculus sound by showing
  that all inference rules are sound with respect to the denotational
  semantics, and all axioms valid (true in every state and interpretation).

  This work is published in \cite{BohrerCPP17} along with a Coq formalization.
  It is based on prior non-mechanized proofs~\cite{DBLP:journals/jar/Platzer16,DBLP:conf/cade/Platzer15}.

\tableofcontents

% sane default for proof documents
\parindent 0pt\parskip 0.5ex

% generated text of all theories
\input{session}

% optional bibliography
\bibliographystyle{abbrv}
\bibliography{root}

\end{document}

%%% Local Variables:
%%% mode: latex
%%% TeX-master: t
%%% End:
