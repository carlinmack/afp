\documentclass[11pt,a4paper]{article}
\usepackage{isabelle,isabellesym}

% further packages required for unusual symbols (see also
% isabellesym.sty), use only when needed

\usepackage{textcomp}
\usepackage{amsmath,amssymb}
  %for \<leadsto>, \<box>, \<diamond>, \<sqsupset>, \<mho>, \<Join>,
  %\<lhd>, \<lesssim>, \<greatersim>, \<lessapprox>, \<greaterapprox>,
  %\<triangleq>, \<yen>, \<lozenge>

%\usepackage{eurosym}
  %for \<euro>

%\usepackage[only,bigsqcap]{stmaryrd}
  %for \<Sqinter>

%\usepackage{eufrak}
  %for \<AA> ... \<ZZ>, \<aa> ... \<zz> (also included in amssymb)

%\usepackage{textcomp}
  %for \<onequarter>, \<onehalf>, \<threequarters>, \<degree>, \<cent>,
  %\<currency>

% this should be the last package used
\usepackage{pdfsetup}

% urls in roman style, theory text in math-similar italics
\urlstyle{rm}
\isabellestyle{it}

% for uniform font size
%\renewcommand{\isastyle}{\isastyleminor}


\begin{document}

\title{The $\lambda\mu$-calculus}
\author{Cristina Matache \and Victor B.~F.~Gomes \and Dominic P.~Mulligan}

\maketitle

\tableofcontents

\abstract{
  The propositions-as-types correspondence is ordinarily presented as linking
  the metatheory of typed $\lambda$-calculi and the proof theory of
  intuitionistic logic. Griffin~\cite{DBLP:conf/popl/Griffin90} observed that
  this correspondence could be extended to classical logic through the use of
  control operators. This observation set off a flurry of further research,
  leading to the development of Parigot’s
  $\lambda\mu$-calculus~\cite{DBLP:conf/lpar/Parigot92}. In this work, we
  formalise $\lambda\mu$-calculus in Isabelle/HOL  and prove several
  metatheoretical properties such as type preservation and progress.
}

% sane default for proof documents
\parindent 0pt\parskip 0.5ex

% generated text of all theories
\input{session}

% optional bibliography
\bibliographystyle{abbrv}
\bibliography{root}

\end{document}

%%% Local Variables:
%%% mode: latex
%%% TeX-master: t
%%% End:
