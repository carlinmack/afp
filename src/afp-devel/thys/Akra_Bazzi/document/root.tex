\documentclass[11pt,a4paper]{article}
\usepackage{amsmath}
\usepackage{isabelle,isabellesym}

% this should be the last package used
\usepackage{pdfsetup}

% urls in roman style, theory text in math-similar italics
\urlstyle{rm}
\isabellestyle{it}


\begin{document}

\title{The Akra--Bazzi theorem and the Master theorem}
\author{Manuel Eberl}
\maketitle

\begin{abstract}
This article contains a formalisation of the Akra--Bazzi method~\cite{akrabazzi} based on a proof by Leighton~\cite{leighton}. It is a generalisation of the well-known Master Theorem for analysing the complexity of Divide \& Conquer algorithms. We also include a generalised version of the Master theorem based on the Akra--Bazzi theorem, which is easier to apply than the Akra--Bazzi theorem itself.

Some proof methods that facilitate applying the Master theorem are also included. For a more detailed explanation of the formalisation and the proof methods, see the accompanying paper (publication forthcoming).
\end{abstract}

\tableofcontents

\parindent 0pt\parskip 0.5ex

\input{session}

\bibliographystyle{abbrv}
\bibliography{root}

\end{document}

%%% Local Variables:
%%% mode: latex
%%% TeX-master: t
%%% End:
