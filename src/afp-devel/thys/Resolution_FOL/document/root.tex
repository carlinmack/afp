\documentclass[11pt,a4paper]{article}
\usepackage[utf8]{inputenc}
\usepackage{isabelle,isabellesym}


% this should be the last package used!
\usepackage{pdfsetup}

% urls in roman style, theory text in math-similar italics
\urlstyle{rm}
\isabellestyle{it}


\begin{document}

\title{The Resolution Calculus for First-Order Logic}
\author{Anders Schlichtkrull}
\maketitle
\begin{abstract}
This theory is a formalization of the resolution calculus for first-order logic. It is proven sound and complete.
The soundness proof uses the substitution lemma, which shows a correspondence between substitutions and updates to
an environment. The completeness proof uses semantic trees, i.e. trees whose paths are partial Herbrand interpretations.
It employs Herbrand's theorem in a formulation which states that an unsatisfiable set of clauses has a finite closed
semantic tree. It also uses the lifting lemma which lifts resolution derivation steps from the ground world up to the
first-order world. The theory is presented in a paper in the Journal of Automated Reasoning \cite{schlichtkrull2018}
which extends a paper presented at the International Conference on Interactive Theorem Proving \cite{schlichtkrull2016}.
An earlier version was presented in an MSc thesis \cite{thesis}. The formalization mostly follows textbooks by Ben-Ari
\cite{ben-ari}, Chang and Lee \cite{chang}, and Leitsch \cite{leitsch}. The theory is part of the IsaFoL project
\cite{isafol}.
\end{abstract}

\tableofcontents

% sane default for proof documents
\parindent 0pt\parskip 0.5ex

% generated text of all theories
\input{session}

% optional bibliography
\bibliographystyle{abbrv}
\bibliography{root}

\end{document}

%%% Local Variables:
%%% mode: latex
%%% TeX-master: t
%%% End:
