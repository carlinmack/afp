\documentclass[11pt,notitlepage,a4paper]{report}
\usepackage{isabelle,isabellesym,eufrak}
\usepackage[english]{babel}

% this should be the last package used
\usepackage{pdfsetup}

% urls in roman style, theory text in math-similar italics
\urlstyle{rm}
\isabellestyle{it}

\input{xy}
\xyoption{curve}
\xyoption{arrow}
\xyoption{matrix}
\xyoption{2cell}
\UseAllTwocells

% Even though I stayed within the default boundary in the JEdit buffer,
% some proof lines wrap around in the PDF document.  To minimize this,
% increase the text width a bit from the default.
\addtolength\textwidth{60pt}
\addtolength\oddsidemargin{-30pt}
\addtolength\evensidemargin{-30pt}

\begin{document}

\title{Category Theory with Adjunctions and Limits}
\author{Eugene W. Stark\\[\medskipamount]
        Department of Computer Science\\
        Stony Brook University\\
        Stony Brook, New York 11794 USA}
\maketitle

\begin{abstract}
This article attempts to develop a usable framework for doing category theory in Isabelle/HOL.
Our point of view, which to some extent differs from that of the previous AFP articles
on the subject, is to try to explore how category theory can be done efficaciously within
HOL, rather than trying to match exactly the way things are done using a traditional
approach.  To this end, we define the notion of category in an ``object-free''
style, in which a category is represented by a single partial composition operation on arrows.
This way of defining categories provides some advantages in the context of HOL, including
the ability to avoid the use of records and the possibility of defining functors and
natural transformations simply as certain functions on arrows, rather than as composite
objects.  We define various constructions associated with the basic notions,
including: dual category, product category, functor category, discrete category, free category,
functor composition, and horizontal and vertical composite of natural transformations.
A ``set category'' locale is defined that axiomatizes the notion ``category of all sets at
a type and all functions between them,'' and a fairly extensive set of properties of set
categories is derived from the locale assumptions.
The notion of a set category is used to prove the Yoneda Lemma in a general setting
of a category equipped with a ``hom embedding,'' which maps arrows of the category
to the ``universe'' of the set category.
We also give a treatment of adjunctions, defining adjunctions via left and right adjoint
functors, natural bijections between hom-sets, and unit and counit natural transformations,
and showing the equivalence of these definitions.
We also develop the theory of limits, including representations of functors,
diagrams and cones, and diagonal functors.  We show that right adjoint functors preserve
limits, and that limits can be constructed via products and equalizers.  We characterize
the conditions under which limits exist in a set category.  We also examine the case of
limits in a functor category, ultimately culminating in a proof that the Yoneda embedding
preserves limits.
\end{abstract}

\tableofcontents

\chapter{Introduction}

This article attempts to develop a usable framework for doing category theory in Isabelle/HOL.
Perhaps the main issue that one faces in doing this is how best to represent what is
essentially a theory of a partially defined operation (composition) in HOL, which is a theory
of total functions.  The fact that in HOL every function is total means that a value must be
given for the composition of any pair of arrows of a category, even if those arrows are not
really composable.  Proofs must constantly concern themselves with whether or not a
particular term does or does not denote an arrow, and whether particular pairs of arrows
are or are not composable.  This kind of issue crops up in the most basic situations,
such as trying to use associativity of composition to prove that two arrows are equal.
Without some sort of systematic way of dealing with this issue, it is hard to do proofs
of interesting results, because one is constantly distracted from the main line of
reasoning by the necessity of proving lemmas that show that various expressions denote
well-defined arrows, that various pairs of arrows are composable, {\em etc.}

In trying to develop category theory in this setting, one notices fairly soon that some
of the problem can be solved by creating introduction rules that allow the proof assistant
to automatically infer, say, that a given term denotes an arrow with a particular
domain and codomain from similar properties of its proper subterms.  This ``upward''
reasoning helps, but it goes only so far.  Eventually one faces a situation in which it is
desired to prove theorems whose hypotheses state that certain terms denote arrows with
particular domains and codomains, but the proof requires similar lemmas about the proper
subterms.  Without some way of doing this ``downward'' reasoning, it becomes very
tedious to establish the necessary lemmas.

Another issue that one faces when trying to formulate category theory within HOL
is the lack of the set-theoretic universe that is usually assumed in traditional
developments.  Since there is no ``type of all sets'' in HOL, one cannot construct
``the'' category {\bf Set} of {\em all} sets and functions between them.
Instead, the best one can do is consider ``a'' category of all sets and functions at
a particular type.  Although the lack of set-theoretic universe would likely cause
complications for some applications of category theory, there are many
applications for which the lack of a universe is not really a hindrance.
So one might well adopt a point of view that accepts {\em a priori} the lack of a
universe and asks instead how much of traditional category theory could be done in
such a setting.

There have been two previous category theory submissions to the AFP.
The first \cite{OKeefe-AFP05} is an exploratory work that develops just enough
category theory to enable the statement and proof of a version of the Yoneda Lemma.
The main features are: the use of records to define categories and functors,
construction of a category of all subsets of a given set, where the arrows are
domain set/codomain set/function triples, and the use of the category
of all sets of elements of the arrow type of category $C$ as the target for the
Yoneda functor for $C$.
The second category theory submission to the AFP \cite{Katovsky-AFP10} is somewhat
more extensive in its scope, and tries to match more closely a traditional development
of category theory through the use of a set-theoretic universe obtained by an
axiomatic extension of HOL.  Categories, functors, and natural transformations
are defined as multi-component records, similarly to \cite{OKeefe-AFP05}.
``The'' category of sets is defined, having as its object and arrow type the type ZF,
which is the axiomatically defined set-theoretic universe.
Included in \cite{Katovsky-AFP10} is a more extensive development of natural
transformations, vertical composition, and functor categories than is to be found in
\cite{OKeefe-AFP05}.  However, as in \cite{OKeefe-AFP05}, the main purely category-theoretic
result in \cite{Katovsky-AFP10} is the Yoneda Lemma.
Beyond the use of ``extensional'' functions, which take on a particular default value
outside of their domains of definition, neither \cite{OKeefe-AFP05} nor \cite{Katovsky-AFP10}
explicitly describe a systematic approach to the problem of obtaining lemmas that
establish when the various terms appearing in a proof denote well-defined arrows.

The present development differs in a number of respects from that of
\cite{OKeefe-AFP05} and \cite{Katovsky-AFP10}, both in style and scope.
The main stylistic features of the present development are as follows:
\begin{itemize}
\item  The notion of a category is defined in an ``object-free'' style,
  motivated by \cite{AHS}, Sec. 3.52-3.53, in which a category is represented by a
  single partial composition operation on arrows.
  This way of defining categories provides some advantages in the context of HOL,
  including the possibility of avoiding extensive use of composite objects constructed
  using records.
  (Katovsky seemed to have had some similar ideas, since he refers in
  \cite{Katovsky-CatThy10} to a theory ``PartialBinaryAlgebra'' that was also motivated
  by \cite{AHS}, although this theory did not ultimately become part of his AFP article.)
\item  Functors and natural transformation are defined simply to be certain
  functions on arrows, where locale predicates are used to express the conditions
  that must be satisfied.  This makes it possible to define functors and natural
  transformations easily using lambda notation without records.
\item  Rules for reasoning about categories, functors, and natural transformations
  are defined so that all ``diagrammatic'' hypotheses reduce to conjunctions of
  assertions, each of which states that a given entity is an arrow, has a particular
  domain or codomain, or inhabits a particular ``hom-set''.  A system of introduction
  and elimination rules is established which permits both ``upward'' reasoning,
  in which such diagrammatic assertions are established for larger terms using corresponding
  assertions about the proper subterms, as well as ``downward'' reasoning, in which diagrammatic
  assertions about proper subterms are inferred from such assertions about a larger
  term, to be carried out automatically.
\item  Constructions on categories, functors, and natural transformations are defined
  using locales in a formulaic fashion.
  As an example, the product category construction is defined using a locale that
  takes two categories (given by their partial composition operations) as parameters.
  The partial composition operation for the product category is given by a function
  ``$comp$'' defined in the locale.  Lemmas proved within the locale include the fact
  that $comp$ indeed defines a category, as well as characterizations of the basic
  notions (domain, codomain, identities, composition) in terms of those of the
  parameter categories.
  For some constructions, such as the product category, it is possible and convenient
  to have a ``transparent'' arrow type, which permits reasoning about the construction
  without having to introduce an elaborate system of constructors, destructors,
  and associated rules.  For other constructions, such as the functor category,
  it is more desirable to use an ``opaque'' arrow type that hides the concrete
  structure, and forces all reasoning to take place using a fixed set of rules.
\item  Rather than commit to a specific concrete construction of a category of sets and
  functions a ``set category'' locale is defined which axiomatizes the properties of the
  category of sets with elements at a particular type and functions between such.
  In keeping with the definitional approach, the axiomatization is shown consistent by
  exhibiting a particular interpretation for the locale, however care is taken to
  to ensure that any proofs making use of the interpretation depend only on the locale
  assumptions and not on the concrete details of the construction.  The set category
  axioms are also shown to be categorical, in the sense that a bijection between the sets
  of terminal objects of two interpretations of the locale extends to an isomorphism of
  categories.  This supports the idea that the locale axioms are an adequate
  characterization of the properties of a category of sets and functions and the details
  of a particular concrete construction can be kept hidden.
\end{itemize}
  
A brief synopsis of the formal mathematical content of the present development is as follows:
\begin{itemize}
\item  Definitions are given for the notions: category, functor, and natural transformation.
\item  Several constructions on categories are given, including: free category,
  discrete category, dual category, product category, and functor category.
\item  Composite functor, horizontal and vertical composite of natural transformations
  are defined, and various properties proved.
\item  The notion of a ``set category'' is defined and a fairly extensive development
  of the consequences of the definition is carried out.
\item  Hom-functors and Yoneda functors are defined and the Yoneda Lemma is proved.
\item  Adjunctions are defined in several ways, including universal arrows,
  natural isomorphisms between hom-sets, and unit and counit natural transformations.
  The relationships between the definitions are established.
\item  The theory of limits is developed, including the notions of diagram, cone, limit cone,
  representable functors, products, and equalizers.  It is proved that a category with
  products at a particular index type has limits of all diagrams at that type.
  The completeness properties of a set category are established.
  Limits in functor categories are explored, culminating in a proof that the Yoneda
  embedding preserves limits.
\end{itemize}

\medskip\par\noindent
{\bf Revision Notes}

The 2018 version of this development was a major revision of the original (2016)
version.  Although the overall organization and content remained essentially the same,
the 2018 version revised the axioms used to define a category, and as a consequence
many proofs required changes.  The purpose of the revision was to obtain a more organized
set of basic facts which, when annotated for use in automatic proof, would yield behavior more
understandable than that of the original version.  In particular, as I gained experience with
the Isabelle simplifier, I was able to understand better how to avoid some of the vexing
problems of looping simplifications that sometimes cropped up when using the original rules.
The new version ``feels'' about as powerful as the original version, or perhaps slightly more so.
However, the new version uses elimination rules in place of some things that were previously
done by simplification rules, which means that from time to time it becomes necessary
to provide guidance to the prover as to where the elimination rules should be invoked.

Another difference between the 2018 version of this document and the original is the
introduction of some notational syntax, which I intentionally avoided in the original.
An important reason for not introducing syntax in the original version was that at the time
I did not have much experience with the notational features of Isabelle, and I was afraid
of introducing hard-to-remove syntax that would make the development more difficult to read
and write, rather than easier.  (I tended to find, for example, that the proliferation of
special syntax introduced in \cite{Katovsky-AFP10} made the presentation seem less readily
accessible than if the syntax had been omitted.)  In the 2018 revision, I introduced
syntax for composition of arrows in a category, and for the notion of ``an arrow inhabiting
a hom-set.''  The notation for composition eases readability by reducing the number of
required parentheses, and the notation for asserting that an arrow inhabits a particular
hom-set gives these assertions a more familiar appearance; making it easier to understand
them at a glance.

This document was revised again in early 2020, prior to the release of Isabelle2020.
That revision incorporated the generic ``concrete category'' construction originally
introduced in \cite{Bicategory-AFP}, and using it systematically as a uniform replacement
for various constructions that were previously done in an {\em ad hoc} manner.
These include the construction of
``functor categories'' of categories of functors and natural transformations,
``set categories'' of sets and functions, and various kinds of free categories.
The awkward ``abstracted category'' construction, which had no interesting mathematical
content but was present in the original version as a solution to a modularity problem that
I no longer deem to be a significant issue, has been removed.
The cumbersome ``horizontal composite'' locale, which was unnecessary given that in this
formalization horizontal composite is given simply by function composition,
has been replaced by a single lemma that does the same job.
Finally, a lemma in the original version that incorrectly advertised itself as being
the ``interchange law'' for natural transformations, has been changed to be the
correct general statement.

The current version of this document incorporates further revisions, made later in 2020
after the release of Isabelle2020.  The theory ``category with pullbacks'',
originally introduced in \cite{Bicategory-AFP}, was moved here and improved somewhat.
In addition, new theories were introduced to cover additional common
situations of categories with certain kinds of limits:
``cartesian category'', which concerns categories with binary products and a terminal object,
``cartesian closed category'', which additionally have exponentials, and
``category with finite limits'', which is shown to be the same as
``category with pullbacks and terminal object''.
To tie things together and to verify the consistency of the locales
(\emph{e.g.}~``cartesian closed category'') for which concrete interpretations have not yet
been given, we construct a category whose objects correspond to the hereditarily finite sets
and whose arrows correspond to functions between such sets, and we show that this category
is cartesian closed and has finite limits.
To facilitate this development, we generalize the ``set category'' construction to cover
some cases in which not every subset of the ``universe'' need determine an object.
In particular, the generalized notion of ``set category'' covers the case in which only
finite sets correspond to objects.  This generalization permits us to treat the category
of hereditarily finite sets as a ``set category'' and to apply some results previously shown
about limits in such a category.

% include generated text of all theories
\input{session}

\bibliographystyle{abbrv}
\bibliography{root}

\end{document}
