\documentclass[11pt,a4paper]{article}
\usepackage[T1]{fontenc}
\usepackage{isabelle,isabellesym}
\usepackage{amsfonts,amsmath,amssymb}
\usepackage{pgfplots}

% this should be the last package used
\usepackage{pdfsetup}

% urls in roman style, theory text in math-similar italics
\urlstyle{rm}
\isabellestyle{it}

\begin{document}

\title{The Lambert $W$ Function on the Reals}
\author{Manuel Eberl}
\maketitle

\begin{abstract}
The Lambert $W$ function is a multi-valued function defined as the inverse function of $x \mapsto x e^x$. Besides numerous applications in combinatorics, physics, and engineering, it also frequently occurs when solving equations containing both $e^x$ and $x$, or both $x$ and $\log x$.

This article provides a definition of the two real-valued branches $W_0(x)$ and $W_{-1}(x)$ and proves various properties such as basic identities and inequalities, monotonicity, differentiability, asymptotic expansions, and the MacLaurin series of $W_0(x)$ at $x = 0$.
\end{abstract}

\tableofcontents
\newpage
\parindent 0pt\parskip 0.5ex

\input{session}

\nocite{corless96}
\bibliographystyle{abbrv}
\bibliography{root}

\end{document}

%%% Local Variables:
%%% mode: latex
%%% TeX-master: t
%%% End:
