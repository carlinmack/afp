\documentclass[10pt,a4paper]{report}
\usepackage[T1]{fontenc}
\usepackage{isabelle,isabellesym}
\usepackage{a4wide}
\usepackage[english]{babel}
\usepackage{eufrak}
\usepackage{amssymb}

% this should be the last package used
\usepackage{pdfsetup}

% urls in roman style, theory text in math-similar italics
\urlstyle{rm}
\isabellestyle{literal}


\begin{document}

\title{An Abstract Formalization of G\"odel's Incompleteness Theorems}
\author{Andrei Popescu \and Dmitriy Traytel}

\maketitle

\begin{abstract} We present an abstract formalization of G\"odel's incompleteness theorems.
We analyze sufficient conditions for the theorems' applicability to a partially specified logic.
Our abstract perspective enables a comparison between alternative approaches from the literature.
These include Rosser's variation of the first theorem, Jeroslow's variation of the second theorem,
and the Swierczkowski--Paulson semantics-based approach. This AFP entry is the main entry point to the results
described in our CADE-27 paper~\cite{DBLP:conf/cade/0001T19}.

\looseness=-1
As part of our abstract formalization's validation, we instantiate our locales twice in the separate
AFP entries \href{https://www.isa-afp.org/entries/Goedel_HFSet_Semantic.html}{Goedel\_HFSet\_Semantic} and
\href{https://www.isa-afp.org/entries/Goedel_HFSet_Semanticless.html}{Goedel\_HFSet\_Semanticless}.
\end{abstract}

\tableofcontents

% sane default for proof documents
\parindent 0pt\parskip 0.5ex

% generated text of all theories
\input{session}

\bibliographystyle{abbrv}
\bibliography{root}

\end{document}

%%% Local Variables:
%%% mode: latex
%%% TeX-master: t
%%% End:
