\documentclass[11pt,a4paper]{book}

\usepackage{isabelle,isabellesym}
\usepackage[english]{babel}
\usepackage{natbib}
\usepackage{times}
\usepackage{amsmath}
\usepackage{amssymb}
\usepackage[only,bigsqcap]{stmaryrd}
\usepackage[T1]{fontenc}
\usepackage[utf8]{inputenc}
\usepackage[all]{xy}
\usepackage{lmodern}
\usepackage[pdftex,colorlinks=true,linkcolor=blue]{hyperref}

% urls in roman style, theory text in math-similar italics
\urlstyle{rm}
\isabellestyle{it}

\begin{document}
\renewcommand{\chapterautorefname}{Chapter}
\renewcommand{\sectionautorefname}{Section}
\renewcommand{\subsectionautorefname}{Section}
\renewcommand{\subsubsectionautorefname}{Section}
\renewcommand{\appendixautorefname}{Appendix}
\renewcommand{\Hfootnoteautorefname}{Footnote}
\newcommand{\lemmaautorefname}{Lemma}
\newcommand{\definitionautorefname}{Definition}

\frontmatter

\title{pGCL for Isabelle}
\author{David Cock}
\maketitle

\tableofcontents

% sane default for proof documents
\parindent 0pt\parskip 0.5ex

\mainmatter

\chapter{Overview}

pGCL is both a programming language and a specification language that
incorporates both probabilistic and nondeterministic choice, in a unified
manner.  Program verification is by \emph{refinement} or \emph{annotation} (or
both), using either Hoare triples, or weakest-precondition entailment, in the
style of GCL \citep{Dijkstra_75}.

This document is divided into three parts: \autoref{c:intro} gives a
tutorial-style introduction to pGCL, and demonstrates the tools provided by
the package; \autoref{c:semantics} covers the development of the semantic
interpretation: \emph{expectation transformers}; and \autoref{c:language}
covers the formalisation of the language primitives, the associated
\emph{healthiness} results, and the tools for structured and automated
reasoning.  This second part follows the technical development of the pGCL
theory package, in detail.  It is not a great place to start learning pGCL.
For that, see either the tutorial or \citet{McIver_M_04}.

This formalisation was first presented (as an overview) in \citet{Cock_12}.
The language has previously been formalised in HOL4 by \citet{Hurd_05}.  Two
substantial results using this package were presented in \citet{Cock_13},
\citet{Cock_14} and \citet{Cock_14a}.

\chapter{Introduction to pGCL}
\label{c:intro}
\input{Primitives}
\input{LoopExamples}
\input{Monty}

\chapter{Semantic Structures}
\label{c:semantics}
\input{Expectations}
\input{Transformers}
\input{Induction}

\chapter{The pGCL Language}
\label{c:language}
\input{Embedding}
\input{Healthiness}
\input{Continuity}
\input{LoopInduction}
\input{Sublinearity}
\input{Determinism}
\input{WellDefined}
\input{Loops}
\input{Algebra}
\input{StructuredReasoning}
\input{Termination}
\input{Automation}

\backmatter

\chapter{Additional Material}
\label{c:additional}
\input{Misc}

\bibliographystyle{plainnat}
\bibliography{root}

\end{document}

%%% Local Variables:
%%% mode: latex
%%% TeX-master: t
%%% End:
