\documentclass[11pt,a4paper]{article}
\usepackage{isabelle,isabellesym}

% further packages required for unusual symbols (see also
% isabellesym.sty), use only when needed

%\usepackage{amssymb}
  %for \<leadsto>, \<box>, \<diamond>, \<sqsupset>, \<mho>, \<Join>,
  %\<lhd>, \<lesssim>, \<greatersim>, \<lessapprox>, \<greaterapprox>,
  %\<triangleq>, \<yen>, \<lozenge>

%\usepackage{eurosym}
  %for \<euro>

%\usepackage[only,bigsqcap]{stmaryrd}
  %for \<Sqinter>

%\usepackage{eufrak}
  %for \<AA> ... \<ZZ>, \<aa> ... \<zz> (also included in amssymb)

%\usepackage{textcomp}
  %for \<onequarter>, \<onehalf>, \<threequarters>, \<degree>, \<cent>,
  %\<currency>

% this should be the last package used
\usepackage{pdfsetup}

% urls in roman style, theory text in math-similar italics
\urlstyle{rm}
\isabellestyle{it}

% for uniform font size
%\renewcommand{\isastyle}{\isastyleminor}


\begin{document}

\title{Spivey's Generalized Recurrence for Bell Numbers}
\author{Lukas Bulwahn}
\maketitle

\begin{abstract}

This entry defines the Bell numbers~\cite{bell-numbers} as the cardinality
of set partitions for a carrier set of given size, and derives Spivey's
generalized recurrence relation for Bell numbers~\cite{spivey-2008}
following his elegant and intuitive combinatorial proof.

As the set construction for the combinatorial proof requires construction of
three intermediate structures, the main difficulty of the formalization is
handling the overall combinatorial argument in a structured way.
The introduced proof structure allows us to compose the combinatorial argument
from its subparts, and supports to keep track how the detailed proof
steps are related to the overall argument. To obtain this structure, this
entry uses set monad notation for the set construction's definition,
introduces suitable predicates and rules, and follows a repeating structure
in its Isar proof.

\end{abstract}

\tableofcontents

% sane default for proof documents
\parindent 0pt\parskip 0.5ex

% generated text of all theories
\input{session}

\nocite{*}

\bibliographystyle{abbrv}
\bibliography{root}

\end{document}

%%% Local Variables:
%%% mode: latex
%%% TeX-master: t
%%% End:
