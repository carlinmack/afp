\documentclass[11pt,a4paper]{article}
\usepackage{isabelle,isabellesym}


% this should be the last package used
\usepackage{pdfsetup}

% urls in roman style, theory text in math-similar italics
\urlstyle{rm}
\isabellestyle{it}

\begin{document}

\title{Ptolemy's Theorem}
\author{Lukas Bulwahn}
\maketitle

\begin{abstract}

This entry provides an analytic proof to Ptolemy's Theorem using
polar form transformation and trigonometric identities.
In this formalization, we use ideas from John Harrison's HOL Light
formalization~\cite{Harrison} and the proof sketch on the Wikipedia entry of
Ptolemy's Theorem~\cite{wiki:PtolemysTheorem-2016}.
This theorem is the 95th theorem of the Top 100 Theorems list~\cite{Wiedijk}.

\end{abstract}

\tableofcontents

% sane default for proof documents
\parindent 0pt\parskip 0.5ex

% generated text of all theories
\input{session}

\bibliographystyle{abbrv}
\bibliography{root}

\end{document}