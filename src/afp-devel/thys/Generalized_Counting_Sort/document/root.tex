\documentclass[11pt,a4paper,fleqn]{article}
\usepackage{isabelle,isabellesym}
\usepackage{amsfonts}
\usepackage{amsmath}
\usepackage{cancel}
\renewcommand{\isastyletxt}{\isastyletext}

% further packages required for unusual symbols (see also
% isabellesym.sty), use only when needed

%\usepackage{amssymb}
  %for \<leadsto>, \<box>, \<diamond>, \<sqsupset>, \<mho>, \<Join>,
  %\<lhd>, \<lesssim>, \<greatersim>, \<lessapprox>, \<greaterapprox>,
  %\<triangleq>, \<yen>, \<lozenge>

%\usepackage{eurosym}
  %for \<euro>

%\usepackage[only,bigsqcap]{stmaryrd}
  %for \<Sqinter>

%\usepackage{eufrak}
  %for \<AA> ... \<ZZ>, \<aa> ... \<zz> (also included in amssymb)

%\usepackage{textcomp}
  %for \<onequarter>, \<onehalf>, \<threequarters>, \<degree>, \<cent>,
  %\<currency>

% this should be the last package used
\usepackage{pdfsetup}

% urls in roman style, theory text in math-similar italics
\urlstyle{rm}
\isabellestyle{it}

% for uniform font size
%\renewcommand{\isastyle}{\isastyleminor}


\begin{document}

\title{An Efficient Generalization of Counting Sort\\for Large, possibly Infinite Key Ranges}
\author{Pasquale Noce\\Software Engineer at HID Global, Italy\\pasquale dot noce dot lavoro at gmail dot com\\pasquale dot noce at hidglobal dot com}
\maketitle

\begin{abstract}
Counting sort is a well-known algorithm that sorts objects of any kind mapped to
integer keys, or else to keys in one-to-one correspondence with some subset of
the integers (e.g. alphabet letters). However, it is suitable for direct use,
viz. not just as a subroutine of another sorting algorithm (e.g. radix sort),
only if the key range is not significantly larger than the number of the objects
to be sorted.

This paper describes a tail-recursive generalization of counting sort making use
of a bounded number of counters, suitable for direct use in case of a large, or
even infinite key range of any kind, subject to the only constraint of being a
subset of an arbitrary linear order. After performing a pen-and-paper analysis
of how such algorithm has to be designed to maximize its efficiency, this paper
formalizes the resulting generalized counting sort (GCsort) algorithm and then
formally proves its correctness properties, namely that (a) the counters' number
is maximized never exceeding the fixed upper bound, (b) objects are conserved,
(c) objects get sorted, and (d) the algorithm is stable.
\end{abstract}

\tableofcontents

% sane default for proof documents
\parindent 0pt\parskip 0.5ex

% generated text of all theories
\input{session}

% bibliography
\bibliographystyle{abbrv}
\bibliography{root}

\end{document}
