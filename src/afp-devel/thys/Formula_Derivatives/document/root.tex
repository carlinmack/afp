\documentclass[11pt,a4paper]{article}
\usepackage{isabelle,isabellesym}

\usepackage{amssymb}

% this should be the last package used
\usepackage{pdfsetup}

% urls in roman style, theory text in math-similar italics
\urlstyle{rm}
\isabellestyle{it}

\begin{document}

\title{Derivatives of Logical Formulas}
\author{Dmitriy Traytel}
\maketitle
\begin{abstract}
  We formalize new decision procedures for WS1S, M2L(Str), and Presburger
  Arithmetics. Formulas of these logics denote regular languages. Unlike
  traditional decision procedures, we do \emph{not} translate formulas into
  automata (nor into regular expressions), at least not explicitly. Instead we
  devise notions of derivatives (inspired by Brzozowski derivatives for regular
  expressions) that operate on formulas directly and compute a syntactic
  bisimulation using these derivatives. The treatment of Boolean connectives and
  quantifiers is uniform for all mentioned logics and is abstracted into a
  locale. This locale is then instantiated by different atomic formulas and
  their derivatives (which may differ even for the same logic under different
  encodings of interpretations as formal words).


  The WS1S instance is described in the draft paper \emph{A Coalgebraic Decision
    Procedure for WS1S}
  \footnote{\url{http://www21.in.tum.de/~traytel/papers/ws1s_derivatives/index.html}}
    by the author.
\end{abstract}

\tableofcontents

% sane default for proof documents
\parindent 0pt\parskip 0.5ex

% generated text of all theories
\input{session}

% optional bibliography
%\bibliographystyle{abbrv}
%\bibliography{root}

\end{document}

%%% Local Variables:
%%% mode: latex
%%% TeX-master: t
%%% End:
