\documentclass[11pt,a4paper]{article}
\usepackage[french,english]{babel}
\usepackage{isabelle,isabellesym}

% this should be the last package used
\usepackage{pdfsetup}

% urls in roman style, theory text in math-similar italics
\urlstyle{rm}
\isabellestyle{tt}

\newcommand{\simpl}{\sloppy \textsc{Simpl}}
\newcommand{\complx}{\sloppy \textsc{Complx}}

\begin{document}

\title{\complx: a Verification Framework for Concurrent Imperative Programs}
\author{Sidney Amani, June Andronick, Maksym Bortin,\\
       Corey Lewis, Christine Rizkallah, Joseph Tuong}
\maketitle

\begin{abstract}  
We propose a concurrency reasoning framework for imperative programs, based on
the Owicki-Gries (OG) foundational shared-variable concurrency method.
Our framework combines the approaches of Hoare-Parallel, a formalisation
of OG in Isabelle/HOL for a simple while-language, and \simpl, a generic
imperative language embedded in Isabelle/HOL, allowing formal reasoning
on C programs.

We define the \complx{} language, extending the syntax and semantics of
\simpl{} with support for parallel composition and synchronisation.
We additionally define an OG logic, which we prove sound
w.r.t. the  semantics, and a verification condition generator, 
both supporting involved low-level imperative constructs such as function
calls and abrupt termination. We illustrate our framework on 
an example that features exceptions, guards and function
calls.
We aim to then target concurrent operating systems, such as the
interruptible eChronos embedded operating system for which we already
have a model-level OG proof using Hoare-Parallel.

\end{abstract}

\tableofcontents

\parindent 0pt\parskip 0.5ex

% generated text of all theories
\input{session}

\end{document}

%%% Local Variables:
%%% mode: latex
%%% TeX-master: t
%%% End:
