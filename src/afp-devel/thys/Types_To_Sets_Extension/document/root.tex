\PassOptionsToPackage{ngerman,main=english}{babel}
\documentclass[11pt,a4paper,fleqn]{report}
\usepackage{iman,extra,isar}
\usepackage{isabelle,isabellesym}
\usepackage{railsetup}

\usepackage[margin={1in,1in}]{geometry}

\usepackage[T1]{fontenc} 
\usepackage{lmodern}
\usepackage{babel}
\usepackage{amsmath}
\usepackage{amssymb}
\usepackage{amsthm}
\usepackage{xspace}
\usepackage{MnSymbol}
\usepackage[utf8]{inputenc}
\usepackage{enumitem}
\usepackage{fontspec}

\usepackage{graphicx}
\usepackage{caption}
\usepackage{proof}
\usepackage{algorithm,algpseudocode}

% bibliography
\usepackage[nottoc]{tocbibind}
\usepackage[square,numbers]{natbib}
\bibliographystyle{abbrvnat}

\usepackage{style}

% this should be the last package used
\usepackage{pdfsetup}

% drop Isabelle tags
\isadroptag{theory}

% enumitem configuration
\setlist{noitemsep,topsep=0pt,parsep=0pt,partopsep=0pt}

% urls in roman style, theory text in math-similar italics
\urlstyle{rm}
\isabellestyle{it}

\begin{document}
\sloppy

\title{
Introduction to the Extension of the Framework 
Types-To-Sets for Isabelle/HOL
} 
\author{Mihails Milehins}
\maketitle

\newpage

\begin{abstract}
In \cite{blanchette_types_2016, kuncar_types_2019}, 
Ondřej Kunčar and Andrei Popescu propose an 
extension of the logic \textit{Isabelle/HOL} and an associated algorithm for the 
relativization of \textit{type-based theorems} to more flexible 
\textit{set-based theorems}, collectively referred to as 
\textit{Types-To-Sets}. One of the aims of their work was to open an 
opportunity for the development of a 
software tool for applied relativization  
in the implementation of the logic 
Isabelle/HOL in the proof assistant \textit{Isabelle}
\cite{paulson_natural_1986}. In this document, 
we provide a prototype of a software framework for the interactive 
automated relativization 
of definitions and theorems in Isabelle/HOL, developed as an extension of 
the proof language \textit{Isabelle/Isar}  
\cite{bertot_isar_1999,wenzel_isabelleisar_2007}.
The software framework incorporates the implementation of the 
proposed extension of the logic and associated tools provided in 
\cite{blanchette_types_2016} and improved further in \cite{immler_smooth_2019} 
by Fabian Immler and Bohua Zhan, and builds upon 
some of the ideas for further work expressed in \cite{immler_smooth_2019}
and \cite{kuncar_types_2019}.
\end{abstract}

\newpage

\renewcommand{\abstractname}{Acknowledgements}
\begin{abstract}
The author would like to acknowledge the assistance that he received from 
the users of the mailing list of Isabelle 
\href{https://lists.cam.ac.uk/mailman/listinfo/cl-isabelle-users}
in the form of answers given to his general queries and
the persons responsible for the development of
Types-To-Sets \cite{blanchette_types_2016, kuncar_types_2019}
and its official extensions 
\cite{immler_smooth_2019, immler_automation_2019}
for providing explanations of the existing functionality of the framework 
and some of the ideas that laid the foundation of this work. Special thanks
go to Fabian Immler: the conceptual design and the layout of the
commands \textbf{tts\_context}, 
\textbf{tts\_lemmas} and \textbf{tts\_lemma} rest largely on his ideas. Special
thanks also go to Andrei Popescu for trying the software and providing feedback. 
Furthermore, the author would like to acknowledge the positive 
impact of \cite{urban_isabelle_2019} and 
\cite{wenzel_isabelle/isar_2019} on his ability to code in Isabelle/ML. 
The author would also like to acknowledge
the positive role that the numerous Q\&A posted on the Stack Exchange network 
(especially Stack Overflow and TeX Stack Exchange) played in the 
development of this work.
Finally, the author would like to express gratitude to all members of his family 
and friends for their continuous support.
\end{abstract}

\newpage

\tableofcontents

\newpage

\parindent 0pt\parskip 0.5ex

\input{session}

\newpage

\bibliography{root}

\end{document}