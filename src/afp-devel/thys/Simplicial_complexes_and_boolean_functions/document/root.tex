\documentclass[11pt,a4paper]{article}
\usepackage{isabelle,isabellesym}

% further packages required for unusual symbols (see also
% isabellesym.sty), use only when needed

%\usepackage{amssymb}
  %for \<leadsto>, \<box>, \<diamond>, \<sqsupset>, \<mho>, \<Join>,
  %\<lhd>, \<lesssim>, \<greatersim>, \<lessapprox>, \<greaterapprox>,
  %\<triangleq>, \<yen>, \<lozenge>

%\usepackage{eurosym}
  %for \<euro>

%\usepackage[only,bigsqcap]{stmaryrd}
  %for \<Sqinter>

%\usepackage{eufrak}
  %for \<AA> ... \<ZZ>, \<aa> ... \<zz> (also included in amssymb)

%\usepackage{textcomp}
  %for \<onequarter>, \<onehalf>, \<threequarters>, \<degree>, \<cent>,
  %\<currency>

%\PassOptionsToPackage{hyphens}{url}\usepackage{hyperref}

% this should be the last package used
\usepackage{pdfsetup}

% urls in roman style, theory text in math-similar italics
\urlstyle{rm}
\isabellestyle{it}

% for uniform font size
%\renewcommand{\isastyle}{\isastyleminor}

\usepackage[english]{babel}
% for \frqq (whatever that actually is)

\begin{document}

\title{Simplicial complexes and Boolean functions\thanks{This research was partially funded by 
    Ministerio de Ciencia e Innovación (Spain),
    grant number PID2020-116641GB-I00.}}
\author{Jesús María Aransay Azofra, Alejandro del Campo López, Julius Michaelis}
\maketitle

\begin{abstract}
  In this work we formalise the isomorphism between 
  simplicial complexes of dimension $n$ and 
  monotone Boolean functions in $n$ variables, 
  mainly following the definitions and results
  as introduced by N. A. Scoville~\cite[Ch. 6]{SC19}.
  We also take advantage of the AFP representation
  of ROBDD (Reduced Ordered Binary Decision 
    Diagrams)~\cite{ROBDD-AFP}
  to compute the ROBDD representation 
  of a given simplicial complex (by means of the isomorphism
  to Boolean functions). Some examples of simplicial complexes 
  and associated Boolean functions are also presented.
\end{abstract}

\tableofcontents

% sane default for proof documents
\parindent 0pt\parskip 0.5ex

\section{Introduction}



% generated text of all theories
\input{session}

% optional bibliography
\bibliographystyle{abbrv}
\bibliography{root}

\end{document}

%%% Local Variables:
%%% mode: latex
%%% TeX-master: t
%%% End:
