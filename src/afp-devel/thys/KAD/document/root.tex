\documentclass[11pt,a4paper]{article}
\usepackage{isabelle,isabellesym,amssymb}

% further packages required for unusual symbols (see also
% isabellesym.sty), use only when needed

%\usepackage{amssymb}
  %for \<leadsto>, \<box>, \<diamond>, \<sqsupset>, \<mho>, \<Join>,
  %\<lhd>, \<lesssim>, \<greatersim>, \<lessapprox>, \<greaterapprox>,
  %\<triangleq>, \<yen>, \<lozenge>

%\usepackage{eurosym}
  %for \<euro>

%\usepackage[only,bigsqcap]{stmaryrd}
  %for \<Sqinter>

%\usepackage{eufrak}
  %for \<AA> ... \<ZZ>, \<aa> ... \<zz> (also included in amssymb)

%\usepackage{textcomp}
  %for \<onequarter>, \<onehalf>, \<threequarters>, \<degree>, \<cent>,
  %\<currency>

% this should be the last package used
\usepackage{pdfsetup}

% urls in roman style, theory text in math-similar italics
\urlstyle{rm}
\isabellestyle{it}

% for uniform font size
%\renewcommand{\isastyle}{\isastyleminor}


\begin{document}

\title{Kleene Algebras with Domain}
\author{Victor B. F. Gomes, Walter Guttmann, Peter H{\"o}fner, \\Georg
  Struth and Tjark Weber}
\maketitle

\begin{abstract}
  Kleene algebras with domain are Kleene algebras endowed with an
  operation that maps each element of the algebra to its domain of
  definition (or its complement) in abstract fashion.  They form a
  simple algebraic basis for Hoare logics, dynamic logics or predicate
  transformer semantics.  We formalise a modular hierarchy of algebras
  with domain and antidomain (domain complement) operations in
  Isabelle/HOL that ranges from domain and antidomain semigroups to
  modal Kleene algebras and divergence Kleene algebras.  We link these
  algebras with models of binary relations and program traces.  We
  include some examples from modal logics, termination and program
  analysis.
\end{abstract}

\tableofcontents

% sane default for proof documents
\parindent 0pt\parskip 0.5ex

\section{Introductory Remarks}

These theory files are intended as a reference formalisation for
variants of Kleene algebras with domain.  The algebraic hierarchy is
developed in a modular way from domain and antidomain semigroups to
modal Kleene algebras in which forward and backward box and diamond
operators interact via conjugations and Galois connections.
Throughout the development we have aimed at readable proofs so that
these theories can be seen as a machine-checked introduction to
reasoning in this setting.  Apart from that, the Isabelle code is only
sparsely annotated, and we refer to a series of articles for further
information.

Our formalisation follows the approaches of Desharnais, Jipsen and
Struth to domain semigroups~\cite{DesharnaisJipsenStruth} and
Desharnais and Struth to families of domain semirings and Kleene
algebras with domain~\cite{DesharnaisStruthSCP,DesharnaisStruthAMAST}.
The link with modal Kleene algebras, Hoare logics and predicate
transformers has been elaborated by M{\"o}ller and
Struth~\cite{MoellerStruth}; a notion of divergence has been added by
Desharnais, M{\"o}ller and Struth~\cite{DesharnaisMoellerStruthLMCS}.
A previous stage of this formalisation has been documented in a
companion article~\cite{guttmannstruthweber11algmeth}.

The target model of these axiomatisations are binary relations, where
the domain operation represents the set of those elements that are
related to some other element.  There is a vast amount of literature
on axiomatising the domain of functions, especially in semigroup
theory.  The deterministic nature of functions, however, leads to
different axiom sets.  An integration of these approaches is left for
future work.

Our Isabelle/HOL formalisation itself is based on a formalisation of
variants of Kleene algebras~\cite{ka}.  An adaptation of Kleene
algebras with domain to the setting of concurrent dynamic
algebra~\cite{FurusawaStruth} can also be found in the Archive of
Formal Proofs~\cite{multirelations}. A formalisation of the original
two-sorted approach to Kleene algebra with
domain~\cite{desharnaismoellerstruth06kad} is left for future work as
well.

% generated text of all theories
\input{session}

% optional bibliography
\bibliographystyle{abbrv}
\bibliography{root}

\end{document}

%%% Local Variables:
%%% mode: latex
%%% TeX-master: t
%%% End:
