\documentclass[11pt,a4paper]{article}
\usepackage{isabelle,isabellesym}

% this should be the last package used
\usepackage{pdfsetup}

% urls in roman style, theory text in math-similar italics
\urlstyle{rm}
\isabellestyle{it}


\begin{document}

\title{Ribbon Proofs for Separation Logic \\ (Isabelle Formalisation)}
\author{John Wickerson}
\maketitle

\begin{abstract}
This document concerns the theory of \emph{ribbon proofs}: a diagrammatic proof system, based on separation logic, for verifying program correctness. We include the syntax, proof rules, and soundness results for two alternative formalisations of ribbon proofs.

Compared to traditional `proof outlines', ribbon proofs emphasise the structure of a proof, so are intelligible and pedagogical. Because they contain less redundancy than proof outlines, and allow each proof step to be checked locally, they may be more scalable. Where proof outlines are cumbersome to modify, ribbon proofs can be visually manoeuvred to yield proofs of variant programs. 
\end{abstract}

\tableofcontents

% sane default for proof documents
\parindent 0pt\parskip 0.5ex

\section{Introduction}

Ribbon proofs are a diagrammatic approach for proving program correctness, based on separation logic. They are due to Wickerson, Dodds and Parkinson~\cite{wickerson+13}, and are also described in Wickerson's PhD dissertation~\cite{wickerson13}. An early version of the proof system, for proving entailments between quantifier-free separation logic assertions, was introduced by Bean~\cite{bean06}.

Compared to traditional `proof outlines', ribbon proofs emphasise the structure of a proof, so are intelligible and pedagogical. Because they contain less redundancy than proof outlines, and allow each proof step to be checked locally, they may be more scalable. Where proof outlines are cumbersome to modify, ribbon proofs can be visually manoeuvred to yield proofs of variant programs. 

In this document, we formalise a two-dimensional graphical syntax for ribbon proofs, provide proof rules, and show that any provable ribbon proof can be recreated using the ordinary rules of separation logic.

In fact, we provide two different formalisations. Our ``stratified'' formalisation sees a ribbon proof as a sequence of rows, with each row containing one step of the proof. This formalisation is very simple, but it does not reflect the visual intuition of ribbon proofs, which suggests that some proof steps can be slid up or down without affecting the validity of the overall proof. Our ``graphical'' formalisation sees a ribbon proof as a graph; specifically, as a directed acyclic nested graph. Ribbon proofs formalised in this way are more manoeuvrable, but proving soundness is trickier, and requires the assumption that separation logic's Frame rule has no side-condition (an assumption that can be validated by using, for instance, variables-as-resource~\cite{bornat+06}).





% generated text of all theories
\input{session}

% optional bibliography
\bibliographystyle{abbrv}
\bibliography{root}

\end{document}

%%% Local Variables:
%%% mode: latex
%%% TeX-master: t
%%% End:
