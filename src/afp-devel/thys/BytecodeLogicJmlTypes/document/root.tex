\documentclass[11pt,a4paper]{article}
\usepackage{isabelle,isabellesym}
\usepackage{amssymb}
\usepackage{stmaryrd}

% this should be the last package used
\usepackage{pdfsetup}

% urls in roman style, theory text in math-similar italics
\urlstyle{rm}
\isabellestyle{it}


\begin{document}

\title{A bytecode logic for JML and types\\ (Isabelle/HOL sources)}
\author{Lennart Beringer and Martin Hofmann}

\maketitle

\begin{abstract}
  This document contains the Isabelle/HOL sources underlying our paper
  \emph{A bytecode logic for JML and
    types}~\cite{DBLP:conf/aplas/BeringerH06}, updated to Isabelle
  2008. We present a program logic for a subset of sequential Java
  bytecode that is suitable for representing both, features found in
  high-level specification language JML as well as interpretations of
  high-level type systems. To this end, we introduce a fine-grained
  collection of assertions, including strong invariants, local
  annotations and VDM-reminiscent partial-correctness specifications.
  Thanks to a goal-oriented structure and interpretation of
  judgements, verification may proceed without recourse to an
  additional control flow analysis. The suitability for interpreting
  intensional type systems is illustrated by the proof-carrying-code
  style encoding of a type system for a first-order functional
  language which guarantees a constant upper bound on the number of
  objects allocated throughout an execution, be the execution
  terminating or non-terminating.

  Like the published paper, the formal development is restricted to a
  comparatively small subset of the JVML, lacking (among other
  features) exceptions, arrays, virtual methods, and static fields.
  This shortcoming has been overcome meanwhile, as our paper has
  formed the basis of the {\sc Mobius} base
  logic~\cite{MobiusDeliverable3.1}, a program logic for the full
  sequential fragment of the JVML. Indeed, the present formalisation
  formed the basis of a subsequent formalisation of the {\sc Mobius}
  base logic in the proof assistant Coq, which includes a proof of
  soundness with respect to the Bicolano operational
  semantics~\cite{Pichardie06}.
\end{abstract}

\tableofcontents

\parindent 0pt\parskip 0.5ex

% include generated text of all theories
\input{session}

\bibliographystyle{abbrv}
\bibliography{root}

\end{document}
