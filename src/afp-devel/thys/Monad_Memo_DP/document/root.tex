\documentclass[11pt,a4paper]{article}
\pagestyle{plain} % turn on page numbers

\usepackage{isabelle,isabellesym}
\usepackage{latexsym}
\usepackage{amssymb}
\usepackage{mathpartir}
%\usepackage{tikz}
%\usepackage{pgfplots}

\usepackage[hidelinks]{hyperref}

% no right margin in quote:
\renewenvironment{quote}
{\list{}{}%
\item\relax}
{\endlist}

\newcommand{\noquotes}[1]{{\renewcommand{\isachardoublequote}{}\renewcommand{\isachardoublequoteopen}{}\renewcommand{\isachardoublequoteclose}{}#1}}

\isabellestyle{it}

\renewcommand{\isacharunderscore}{\_}
\renewcommand{\isacharunderscorekeyword}{\_}
\renewcommand{\isadigit}[1]{{\rm #1}}

% for uniform font size
\renewcommand{\isastyle}{\isastyleminor}

\newcommand{\eqnum}{\refstepcounter{equation}\hfill(\theequation)}

\hyphenation{Isa-belle}


\begin{document}

\title{Monadification, Memoization\\ and Dynamic Programming}
\author{Simon Wimmer \and Shuwei Hu \and Tobias Nipkow}
\date{Technical University of Munich\\[\baselineskip] \today}
\maketitle

\begin{abstract}
We present a lightweight framework for the
automatic  verified  (functional  or  imperative)  memoization  of  recursive functions.
Our tool can turn a pure Isabelle/HOL function definition into a monadified version
in a state monad or the Imperative HOL heap monad, and prove a correspondence theorem.
We provide a variety of memory implementations for the two types of monads.
A number of simple techniques allow us to achieve bottom-up computation
and space-efficient memoization.
The framework's utility is demonstrated on a number of representative dynamic programming problems.
A detailed description of our work can be found in the accompanying paper \cite{DP-ITP-2018}.
\end{abstract}

\tableofcontents

\pagebreak

% sane default for proof documents
\parindent 0pt\parskip 0.5ex

\input{session.tex}

%\bibliographystyle{splncs03}
\bibliographystyle{abbrv}
\bibliography{root}

\end{document}
