\documentclass[11pt,a4paper]{article}
\usepackage{isabelle,isabellesym}

% this should be the last package used
\usepackage{pdfsetup}

% urls in roman style, theory text in math-similar italics
\urlstyle{rm}
%\isabellestyle{it}


\begin{document}

\title{Graph Saturation}
\author{Sebastiaan J. C. Joosten}
\maketitle

\begin{abstract}
  This is an Isabelle/HOL formalisation of graph saturation, closely following a paper by the author on graph saturation~\cite{Joosten18}.
  Nine out of ten lemmas of the original paper are proven in this formalisation.
  The formalisation additionally includes two theorems that show the main premise of the paper: that consistency and entailment are decided through graph saturation.
  This formalisation does not give executable code, and it did not implement any of the optimisations suggested in the paper.
\end{abstract}

\tableofcontents

\section{Introduction}
Although the formalisation follows a paper by the author on graph saturation~\cite{Joosten18}, it is foremost a formalisation.
This document highlights the differences, where applicable.
Nevertheless, the reader is advised to start by reading~\cite{Joosten18}.
A copy might be available on \url{http://sjcjoosten.nl/4-publications/joosten18/}.

The first publication of this graph saturation algorithm is in \cite{Joosten17a}.
While that paper contains a somewhat more category-theoretical view,
it also has fewer proofs and less rigor.
Graph Saturation was originally developed to potentially benefit the Ampersand compiler~\cite{Michels11}.

% include generated text of all theories
\input{session}

\paragraph*{acknowledgements}
We thank Gerwin Klein for making an example submission in the AFP~\cite{Klein04}, which was of great help in making this submission.

\bibliographystyle{abbrv}
\bibliography{root}

\end{document}
