\RequirePackage{luatex85}

\documentclass[11pt,notitlepage,a4paper]{report}
\usepackage{isabelle,isabellesym,eufrak}
\usepackage[english]{babel}

% For graphics files
\usepackage{graphicx}

% this should be the last package used
\usepackage{pdfsetup}

% urls in roman style, theory text in math-similar italics
\urlstyle{rm}
\isabellestyle{it}

% XYPic package, for drawing commutative diagrams.
\input{xy}
\xyoption{curve}
\xyoption{arrow}
\xyoption{matrix}
\xyoption{2cell}
\xyoption{line}
\UseAllTwocells

% Even though I stayed within the default boundary in the JEdit buffer,
% some proof lines wrap around in the PDF document.  To minimize this,
% increase the text width a bit from the default.
\addtolength\textwidth{60pt}
\addtolength\oddsidemargin{-30pt}
\addtolength\evensidemargin{-30pt}

\begin{document}

\title{Bicategories}
\author{Eugene W. Stark\\[\medskipamount]
        Department of Computer Science\\
        Stony Brook University\\
        Stony Brook, New York 11794 USA}
\maketitle

\begin{abstract}
Taking as a starting point the author's previous work
(\cite{Category3-AFP} \cite{MonoidalCategory-AFP})
on developing aspects of category theory in Isabelle/HOL, this article gives a
compatible formalization of the notion of ``bicategory'' and develops a
framework within which formal proofs of facts about bicategories can be given.
The framework includes a number of basic results, including the
Coherence Theorem, the Strictness Theorem, pseudofunctors and biequivalence,
and facts about internal equivalences and adjunctions in a bicategory.
As a driving application and demonstration of the utility of the framework,
it is used to give a formal proof of a theorem, due to Carboni, Kasangian,
and Street \cite{carboni-et-al}, that characterizes up to biequivalence
the bicategories of spans in a category with pullbacks.
The formalization effort necessitated the filling-in of many details
that were not evident from the brief presentation in the original paper,
as well as identifying a few minor corrections along the way.

Revisions made subsequent to the first version of this article added additional
material on pseudofunctors, pseudonatural transformations, modifications, and equivalence
of bicategories; the main thrust being to give a proof that a pseudofunctor is
a biequivalence if and only if it can be extended to an equivalence of bicategories.
\end{abstract}

\tableofcontents

\phantomsection
\addcontentsline{toc}{chapter}{Introduction}
\chapter*{Introduction}

Bicategories, introduced by B\'{e}nabou \cite{benabou}, are a generalization of categories
in which the sets of arrows between pairs of objects (\emph{i.e.}~the ``hom-sets'')
themselves have the structure of categories.  In a typical formulation, the definition of
bicategories involves three separate kinds of entities: \emph{objects} (or \emph{$0$-cells}),
\emph{arrows} (or \emph{$1$-cells}), and morphisms between arrows (or \emph{$2$-cells}).
There are two kinds of composition: \emph{vertical} composition, which composes $2$-cells
within a single hom-category, and \emph{horizontal} composition, which composes $2$-cells in
``adjacent'' hom-categories ${\rm hom}(A, B)$ and ${\rm hom}(B, C)$.
Horizontal composition is required to be functorial with respect to vertical composition;
the identification of a $1$-cell with the corresponding identity $2$-cell then leads to the
ability to horizontally compose $1$-cells with $2$-cells (\emph{i.e.}~``whiskering'')
and to horizontally compose $1$-cells with each other.
Each hom-category ${\rm hom}(A, A)$ is further equipped with an \emph{identity} $1$-cell
${\rm id}_A$, which serves as a unit for horizontal composition.
In a \emph{strict} bicategory, also known as a \emph{$2$-category}, the usual unit and
associativity laws for horizontal composition are required to hold exactly,
or (as it is said) ``on the nose''.
In a general bicategory, these laws are only required to hold ``weakly'';
that is, up to a collection of (vertical) isomorphisms that satisfy certain
\emph{coherence conditions}.
A bicategory, all of whose hom-categories are discrete, is essentially an ordinary category.
A bicategory with just one object amounts to a monoidal category whose tensor is given by
horizontal composition.
Alternatively, we may think of bicategories as a generalization of monoidal categories in
which the tensor is permitted to be a partial operation, in analogy to the way in which
ordinary categories can be considered as a generalization of monoids.

A standard example of a bicategory is \textbf{Cat}, the bicategory whose $0$-cells are
categories, whose $1$-cells are functors, and whose $2$-cells are natural transformations.
This is in fact a $2$-category; however, as two categories that are related by an equivalence
of categories have the same ``categorical'' properties, it is often more sensible to
consider constructions on categories as given up to equivalence, rather than up to
isomorphism, and this leads to considering \textbf{Cat} as a bicategory and using
bicategorical constructions rather than as a $2$-category and using $2$-categorical ones.
This is one reason for the importance of bicategories: as Street \cite{street-fibrations-ii} remarks,
``In recent years it has become even more obvious that, although the fundamental constructions
of set theory are categorical, the fundamental constructions of category theory are bicategorical.''

An alternative reason for studying bicategories, which is more aligned with my own
personal interests and forms a major reason why I chose to pursue the present project,
is that they provide an elegant framework for theories of generalized relations,
as has been shown by Carboni, Walters, Street, and others \cite{carboni-et-al}
\cite{cartesian-bicategories-i} \cite{cartesian-bicategories-ii} \cite{carboni-partial-maps}.
Indeed, the category of sets and relations becomes a bicategory by taking the inclusions
between relations as $2$-cells and thereby becomes an exemplar of the notion
bicategory of relations which itself is a specialization of the notion of
cartesian bicategory \cite{cartesian-bicategories-i} \cite{cartesian-bicategories-ii}.
In the study of the semantics of programming languages containing nondeterministic or
concurrent constructs, it is natural to consider the meaning of a program in such a language
as some kind of relation between inputs and outputs.  Ordinary relations can be used for
this purpose in simple situations, but they fail to be adequate for the study of higher-order
nondeterministic programs or for concurrent programs that engage in interaction with their environment,
so some sort of notion of generalized relation is needed.  One is therefore led to try to identify
some kind of bicategories of generalized relations as framework suitable for defining the
semantics of such programs.  One expects these to be instances of cartesian bicategories.

I attempted for a long time to try to develop a semantic framework for a certain class of
interactive concurrent programs along the lines outlined above, but ultimately failed to obtain
the kind of comprehensive understanding that I was seeking.  The basic idea was to try to
regard a program as denoting a kind of generalized machine, expressed as some sort of
bimodule or two-sided fibration ({\em cf.}~\cite{street-fibrations-i} \cite{street-fibrations-ii}),
to be represented as a certain kind of span in an underlying category of ``maps'',
which would correspond to the meanings of deterministic programs.
A difficulty with trying to formulate any kind of theory like this is that there quickly gets
to be a lot of data and a lot of properties to keep track of, and it was certainly more than
I could handle.
For example, bicategories have objects, $1$-cells, and $2$-cells, as well as domains, codomains,
composition and identities for both the horizontal and vertical structure.
In addition, there are unit and associativity isomorphisms for the weak horizontal composition,
as well as their associated coherence conditions.
Cartesian bicategories are symmetric monoidal bicategories, which means that there is an additional
tensor product, which comes with another set of canonical isomorphisms and coherence conditions.
Still more canonical morphisms and coherence conditions are associated with the cartesian structure.
Even worse, in order to give a proper account of the computational ideas I was hoping to capture,
the underlying category of maps would at least have to be regarded as an ordered category,
if not a more general $2$-category or bicategory, so the situation starts to become truly daunting.

With so much data and so many properties, it is unusual in the literature to find proofs written
out in anything approaching complete detail.
To the extent that proofs are given, they often involve additional assumptions made purely for
convenience and presentational clarity, such as assuming that the bicategories under consideration
are strict when actually they are not, and then discharging these assumptions by appeals to informal
arguments such as ``the result holds in the general case because we can always replace a non-strict
bicategory by an equivalent strict one.''
This is perhaps fine if you happen to have finely honed insight, but in my case I am always left
wondering if something important hasn't been missed or glossed over, and I don't trust very much
my own ability to avoid gross errors if I were to work at the same level of detail as the proofs
that I see in the literature.
So my real motivation for the present project was to try to see whether a proof assistant
would actually be useful in carrying out fully formalized, machine-checkable proofs of some kind
of interesting facts about bicategories.  I also hoped in the process to develop a better
understanding of some concepts that I knew that I hadn't understood very well.

The project described in the present article is divided into two main parts.
The first part, which comprises Chapter 1, seeks to develop a formalization of the notion of
bicategory using Isabelle/HOL and to prove various facts about bicategories that are required
for a subsequent application.  Additional goals here are:
(1) to be able to make as much use as possible of the formalizations previously created for
categories \cite{Category3-AFP} and monoidal categories \cite{MonoidalCategory-AFP};
(2) to create a plausibly useful framework for future extension; and
(3) to better understand some subtleties involved in the definition of bicategory.
In this chapter, we give an HOL formalization of bicategories that makes use of and extends the
formalization of categories given in \cite{Category3-AFP}.  In that previous work, categories
were formalized in an ``object-free'' style in terms of a suitably defined associative partial
binary operation of composition on a single type.  Elements of the type that behave as units
for the composition were called ``identities'' and the ``arrows'' were identified as
the elements of the type that are composable both on the left and on the right with identities.
The identities composable in this way with an arrow were then shown to be uniquely determined,
which permitted domain and codomain functions to be defined.
This formalization of categories is economical in terms of basic data (only a single partial
binary operation is required), but perhaps more importantly, functors and natural transformations
need not be defined as structured objects, but instead can be taken to be ordinary functions
between types that suitably preserve arrows and composition.

In order to carry forward unchanged the framework developed for categories, for the
formalization of bicategories we take as a jumping-off point the somewhat offbeat view of
a bicategory as a single global category under vertical composition (the arrows are
the $2$-cells), which is then equipped with an additional partial binary operation of
horizontal composition.  This point of view corresponds to thinking of bicategories as
generalizations of monoidal categories in which the tensor is allowed to be a partial
operation.  In a direct generalization of the approach taken for categories,
we then show that certain \emph{weak units} with respect to the horizontal composition play
the role of $0$-cells (the identities with respect to vertical composition play the role
of $1$-cells) and that we can define the \emph{sources} and \emph{targets} of an arrow
as the sets of weak units horizontally composable on the right and on the left with it.
We then define a notion of weak associativity for the horizontal composition and arrive
at the definition of a \emph{prebicategory}, which consists of a (vertical) category equipped
with an associative weak (horizontal) composition, subject to the additional assumption
that every vertical arrow has a nonempty set of sources and targets with respect to
the horizontal composition.
We then show that, to obtain from a prebicategory a structure that satisfies a more
traditional-looking definition of a bicategory, all that is necessary is to choose
arbitrarily a particular representative source and target for each arrow.
Moreover, every bicategory determines a prebicategory by simply forgetting the chosen
sources and targets.
This development clarifies that an \emph{a priori} assignment of source and target objects
for each $2$-cell is merely a convenience, rather than an element essential to the notion
of bicategory.

Additional highlights of Chapter 1 are as follows:
\begin{itemize}
\item  As a result of having formalized bicategories essentially as ``monoidal categories with
  partial tensor'', we are able to generalize to bicategories, in a mostly straightforward way,
  the proof of the Coherence Theorem we previously gave for monoidal categories in
  \cite{MonoidalCategory-AFP}.
  We then develop some machinery that enables us to apply the Coherence Theorem to shortcut
  certain kinds of reasoning involving canonical isomorphisms.
%
\item  Using the syntactic setup developed for the proof of the Coherence Theorem, we also
  give a proof of the Strictness Theorem, which states that every bicategory is biequivalent
  to a $2$-category, its so-called ``strictification''.
%
\item  We define the notions of internal equivalence and internal adjunction in a bicategory
  and prove a number of basic facts about these notions, including composition of equivalences
  and adjunctions, and that every equivalence can be refined to an adjoint equivalence.
%
\item  We formalize the notion of a pseudofunctor between bicategories, generalizing the
  notion of a monoidal functor between monoidal categories and we show that pseudofunctors
  preserve internal equivalences and adjunctions.
%
\item  We define a sub-class of pseudofunctors which we call \emph{equivalence pseudofunctors}.
  Equivalence pseudofunctors are intended to coincide with those pseudofunctors that can
  be extended to an equivalence of bicategories, but we do not attempt to give an independent
  definition equivalence of bicategories in the present development.  Instead, we establish various
  properties of equivalence pseudofunctors to provide some confidence that the notion has been
  formalized correctly.  Besides establishing various preservation results, we prove that,
  given an equivalence pseudofunctor, we may obtain one in the converse direction.
  For the rest of this article we use the property of two bicategories being connected by an
  equivalence pseudofunctor as a surrogate for the property of biequivalence,
  leaving for future work a more proper formulation of equivalence of bicategories and a
  full verification of the relationship of this notion with equivalence pseudofunctors.
\end{itemize}

The second part of the project, presented in Chapter 2, is to demonstrate the utility of
the framework by giving a formalized proof of a nontrivial theorem about bicategories.
For this part, I chose to tackle a theorem of Carboni, Kasangian, and Street
(\cite{carboni-et-al}, ``CKS'' for short)
which gives axioms that characterize up to equivalence those bicategories whose $1$-cells are
spans of arrows in an underlying category with pullbacks and whose $2$-cells are arrows
of spans.  The original paper is very short (nine pages in total) and the result I planned to
formalize (Theorem 4) was given on the sixth page.  I thought I had basically understood this result
and that the formalization would not take very long to accomplish, but I definitely
underestimated both my prior understanding of the result and the amount of auxiliary material
that it would be necessary to formalize before I could complete the main proof.
Eventually I did complete the formalization, and in the process filled in what seemed to me
to be significant omissions in Carboni, Kasangian, and Street's presentation, as well as
correcting some errors of a minor nature.

Highlights of Chapter 2 are the following:
\begin{itemize}
\item  A formalization of the notion of a category with chosen pullbacks, a proof that
  this formalization is in agreement with the general definition of limits we gave
  previously in \cite{Category3-AFP}, and the development of some basic properties
  of a category with pullbacks.
%
\item  A construction, given a category $C$ with chosen pullbacks, of the ``span bicategory''
  ${\rm Span}(C)$, whose objects are those of the given category, whose $1$-cells are spans
  of arrows of $C$, and whose $2$-cells are arrows of spans.
  We characterize the maps (the \emph{i.e.}~left adjoints) in ${\rm Span}(C)$ as
  exactly those spans whose ``input leg'' is invertible.
%
\item  A formalization of the notion of \emph{tabulation} of a $1$-cell in a bicategory
  and a development of some of its properties.  Tabulations are a kind of bicategorical
  limit introduced by CKS, which can be used to define a kind of biuniversal way of factoring
  a $1$-cell up to isomorphism as the horizontal composition of a map and the adjoint of
  a map.
%
\item  A formalization of \emph{bicategories of spans}, which are bicategories that satisfy
  three axioms introduced in CKS.  We give a formal proof of CKS Theorem 4,
  which characterizes the bicategories of spans as those bicategories that are biequivalent
  to a bicategory ${\rm Span}(C)$ for some category $C$ with pullbacks.
  One direction of the proof shows that if $C$ is a category with pullbacks,
  then ${\rm Span}(C)$ satisfies the axioms for a bicategory of spans.
  Moreover, we show that the notion ``bicategory of spans'' is preserved under equivalence
  of bicategories, so that in fact any bicategory biequivalent to one of the form ${\rm Span}(C)$
  is a bicategory of spans.
  Conversely, we show that if $B$ is a bicategory of spans, then $B$ is biequivalent
  to ${\rm Span}({\rm Maps}(B))$, where ${\rm Maps}(B)$ is the so-called \emph{classifying category}
  of the maps in $B$, which has as objects those of $B$ and as arrows the isomorphism classes
  of maps in $B$.

  In order to formalize the proof of this result, it was necessary to develop a number of
  details not mentioned by CKS, including ways of composing tabulations vertically and
  horizontally, and spelling out a way to choose pullbacks in ${\rm Maps}(B)$ so that
  the tupling of arrows of ${\rm Maps}(B)$ obtained using the chosen pullbacks agrees
  with that obtained through horizontal composition of tabulations.
  These details were required in order to give the definition of the compositor for an equivalence
  pseudofunctor ${\rm SPN}$ from $B$ to ${\rm Span}({\rm Maps}(B))$ and establish the
  necessary coherence conditions.
\end{itemize}

In the end, I think it can be concluded that Isabelle/HOL can be used with benefit to formalize
proofs about bicategories.  It is certainly very helpful for keeping track of the data
involved and the proof obligations required.  For example, in the formalization given here,
a total of 99 separate subgoals are involved in proving that a given set of data constitutes
a bicategory (only 7 subgoals are required for an ordinary category)
and another 29 subgoals must be proved in order to establish a pseudofunctor between two
bicategories (only 5 additional subgoals are required for an ordinary functor),
but the proof assistant assumes the burden of keeping track of these proof obligations and
presenting them to the human user in a structured, understandable fashion.
On the other hand, some of the results proved here still required some lengthy equational
``diagram chases'' for which the proof assistant (at least so far) didn't provide that much help
(aside from checking their correctness).
An exception to this was in the case of equational reasoning about expressions constructed
purely of canonical isomorphisms, which our formulation of the Coherence Theorem permitted
to be carried out automatically by the simplifier.
It seems likely, though, that there is still room for more general procedures to be developed
in order to allow other currently lengthy chains of equational reasoning to be carried out
automatically.

\medskip\par\noindent
{\bf Revision Notes}

The original version of this article dates from January, 2020.
The current version of this article incorporates revisions made throughout 2020.
A number of the changes made in early to mid-2020 consisted of minor improvements
and speedups.  A more major change made in this period was that the theory
``category with pullbacks'' was moved to \cite{Category3-AFP}, where it more
logically belongs.
In late 2020 additional material was added relating to pseudofunctors,
pseudonatural transformations, and equivalence of bicategories.
The main result shown was that a pseudofunctor is a biequivalence if and only
if it can be extended to an equivalence of bicategories.
This important result was sidestepped in the original version of this article,
but the author felt that it was a glaring omission that should be corrected.
Unfortunately, to formalize these results required some rather lengthy calculations
in order to establish coherence conditions.  These calculations added significantly
to the line count of this article, as well as the time and memory required to
validate the proofs.

\phantomsection
\addcontentsline{toc}{chapter}{Preliminaries}
\chapter*{Preliminaries}

\input{IsomorphismClass.tex}

\chapter{Bicategories}

\input{Prebicategory.tex}
\input{Bicategory.tex}
\input{Coherence.tex}
\input{CanonicalIsos.tex}
\input{Subbicategory.tex}
\input{InternalEquivalence.tex}
\input{Pseudofunctor.tex}
\input{Strictness.tex}
\input{InternalAdjunction.tex}
\input{PseudonaturalTransformation.tex}
\input{Modification.tex}
\input{EquivalenceOfBicategories.tex}

\chapter{Bicategories of Spans}

\input{SpanBicategory.tex}
\input{Tabulation.tex}
\input{BicategoryOfSpans.tex}

\phantomsection
\addcontentsline{toc}{chapter}{Bibliography}

\bibliographystyle{abbrv}
\bibliography{root}

\end{document}
