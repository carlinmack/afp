\documentclass[11pt,a4paper]{article}
\usepackage{isabelle,isabellesym}

% this should be the last package used
\usepackage{pdfsetup}

% urls in roman style, theory text in math-similar italics
\urlstyle{rm}
\isabellestyle{it}


\begin{document}

\title{Abstract Completeness}
\author{Jasmin Christian Blanchette, Andrei Popescu, and Dmitriy Traytel}

\maketitle

\begin{abstract}
  This is a formalization of an abstract property of possibly infinite
  derivation trees (modeled by a codatatype), that represents the core of a
  Beth--Hintikka-style proof of the first-order logic completeness theorem and
  is independent of the concrete syntax or inference rules. This work is
  described in detail in a publication by the authors \cite{bla-compl}.

  The abstract proof can be instantiated for a wide range of Gentzen and tableau
  systems as well as various flavors of FOL---e.g., with or without predicates,
  equality, or sorts. Here, we give only a toy example instantiation with
  classical propositional logic. A more serious instance---many-sorted FOL with
  equality---is described elsewhere \cite{bla-mech}.
\end{abstract}

\bibliographystyle{abbrv}
\bibliography{root}

\tableofcontents

% sane default for proof documents
\parindent 0pt\parskip 0.5ex

% generated text of all theories
\input{session}

\end{document}

%%% Local Variables:
%%% mode: latex
%%% TeX-master: t
%%% End:
