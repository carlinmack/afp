\PassOptionsToPackage{ngerman,main=english}{babel}
\documentclass[11pt,a4paper]{article}
\usepackage{iman,extra,isar}
\usepackage{isabelle,isabellesym}
\usepackage{railsetup}

\usepackage[margin={1in,1in}]{geometry}

\usepackage[T1]{fontenc} 
\usepackage{lmodern}
\usepackage{babel}
\usepackage{amsmath}
\usepackage{amssymb}
\usepackage{amsthm}
\usepackage{xspace}
\usepackage{MnSymbol}
\usepackage[utf8]{inputenc}
\usepackage{enumitem}
\usepackage{fontspec}

\usepackage{graphicx}
\usepackage{proof}

% bibliography
\usepackage[nottoc]{tocbibind}
\usepackage[square,numbers]{natbib}
\bibliographystyle{abbrvnat}

% this should be the last package used
\usepackage{pdfsetup}

% drop Isabelle tags
\isadroptag{theory}

% enumitem configuration
\setlist{noitemsep,topsep=0pt,parsep=0pt,partopsep=0pt}

% urls in roman style, theory text in math-similar italics
\urlstyle{rm}
\isabellestyle{it}

\begin{document}
\sloppy

\title{Conditional Simplification} 
\author{Mihails Milehins}
\maketitle

\newpage

\begin{abstract}
The document presents a reference manual for the 
framework Conditional Simplification: 
a collection of experimental general-purpose methods for the object logic
Isabelle/HOL (e.g., see \cite{yang_comprehending_2017}) 
of the formal proof assistant Isabelle
\cite{paulson_natural_1986}.
The methods that are provided in the collection offer the functionality 
that is similar to certain aspects of the functionality provided by the 
standard proof methods of Isabelle that combine classical reasoning 
and simplification, but use a different approach for 
rewriting/simplification. More specifically, the methods provided in the
collection allow for the side conditions of
the rewrite rules to be solved via intro-resolution.
\end{abstract}

\newpage

\renewcommand{\abstractname}{Acknowledgements}
\begin{abstract}

The author would like to acknowledge the assistance that he received from 
the users of the mailing list of Isabelle 
\href{https://lists.cam.ac.uk/mailman/listinfo/cl-isabelle-users}
in the form of answers given to his general queries. 

Furthermore, the author would like to acknowledge the positive 
impact of \cite{urban_isabelle_2019} and 
\cite{wenzel_isabelle/isar_2019} on his ability to code in Isabelle/ML.
Moreover, the author would like to acknowledge
the positive role that numerous Q\&A posted on the Stack Exchange network 
(especially Stack Overflow and TeX Stack Exchange) played in the 
development of this work. 

The author would also like to express gratitude to all members of his family 
and friends for their continuous support.

\end{abstract}

\newpage

\tableofcontents

\newpage

\parindent 0pt\parskip 0.5ex

\input{session}

\newpage

\bibliography{root}

\end{document}