\documentclass[11pt,a4paper]{article}
\usepackage[T1]{fontenc}
\usepackage{isabelle,isabellesym}
\usepackage{xspace}

% further packages required for unusual symbols (see also
% isabellesym.sty), use only when needed
\usepackage{url}
\usepackage{amssymb}
  %for \<leadsto>, \<box>, \<diamond>, \<sqsupset>, \<mho>, \<Join>,
  %\<lhd>, \<lesssim>, \<greatersim>, \<lessapprox>, \<greaterapprox>,
  %\<triangleq>, \<yen>, \<lozenge>

% this should be the last package used
\usepackage{pdfsetup}

% urls in roman style, theory text in math-similar italics
\urlstyle{rm}
\isabellestyle{it}

% for uniform font size
\renewcommand{\isastyle}{\isastyleminor}

\renewcommand{\isamarkupchapter}[1]{\section{#1}}
\renewcommand{\isamarkupsection}[1]{\subsection{#1}}
\renewcommand{\isamarkupsubsection}[1]{\subsubsection{#1}}
\renewcommand{\isamarkupsubsubsection}[1]{\paragraph{#1}}

\begin{document}

\title{Orbit-Stabiliser Theorem with Application to Rotational Symmetries}
\author{Jonas Rädle}

\maketitle
\begin{abstract}
The Orbit-Stabiliser theorem is a simple result in the algebra of groups that factors
the order of a group into the sizes of its orbits and stabilisers. 

We formalize the notion of a group action and the related concepts of orbits and stabilisers.
This allows us to prove the orbit-stabiliser theorem.

In the second part of this work, we formalize the tetrahedral group and use the orbit-stabiliser 
theorem to prove that there are twelve (orientation-preserving) rotations of the tetrahedron.

\end{abstract}

\setcounter{tocdepth}{2}
\tableofcontents
\newpage

% sane default for proof documents
\parindent 0pt\parskip 0.5ex

% generated text of all theories
\input{session}

% optional bibliography
\bibliographystyle{plain}
\bibliography{root}

\end{document}

%%% Local Variables:
%%% mode: latex
%%% TeX-master: t
%%% End:
