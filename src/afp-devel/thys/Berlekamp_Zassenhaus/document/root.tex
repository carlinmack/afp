\documentclass[11pt,a4paper]{article}
\usepackage{isabelle,isabellesym}

\usepackage{amssymb}
\usepackage{amsmath}
\usepackage[ruled,noend]{algorithm2e}
\DontPrintSemicolon

\usepackage{xspace}

% this should be the last package used
\usepackage{pdfsetup}

% urls in roman style, theory text in math-similar italics
\urlstyle{rm}
\isabellestyle{it}

\newcommand\isafor{\textsf{IsaFoR}}
\newcommand\ceta{\textsf{Ce\kern-.18emT\kern-.18emA}}
\newcommand\rats{\mathbb{Q}}
\newcommand\ints{\mathbb{Z}}
\newcommand\reals{\mathbb{R}}
\newcommand\complex{\mathbb{C}}

\newcommand\GFpp[1]{\ensuremath{\text{GF}(#1)}}
\newcommand\GFp{\GFpp{p}}
\newcommand\ring[1][p^k]{\ensuremath{\ints/{#1}\ints}\xspace}
\newcommand\tint{\isa{int}}
\newcommand\tlist{\isa{list}}
\newcommand\tpoly{\isa{poly}}
\newcommand\tto{\Rightarrow}
\newcommand\sqfree{\isa{square\_free}\xspace}
\newcommand\assumes{\isakeyword{assumes}\xspace}
\newcommand\idegree{\isa{degree}}
\newcommand\iand{\isakeyword{and}\xspace}
\newcommand\shows{\isakeyword{shows}}
\newcommand\bz{\isa{berlekamp\_zassenhaus\_factorization}\xspace}
\newcommand\fs{\mathit{fs}}
\newcommand\listprod{\isa{prod\_list}}
\newcommand\set{\isa{set}}
\newcommand\irred{\isa{irreducible}}
\newcommand\rTH[1]{Theorem~\ref{#1}}



\makeatletter
\protected\def\myDot{%
 \@ifnextchar.{}{.%
  \@ifnextchar,{}{%
   \@ifnextchar:{}{%
    \@ifnextchar;{}{%
     \@ifnextchar~{}{\ }
}}}}}
\makeatother

\newcommand\etal{{et~al}\myDot}


\newtheorem{theorem}{Theorem}

\begin{document}

\title{The Factorization Algorithm of Berlekamp and Zassenhaus \footnote{Supported by FWF (Austrian Science Fund) project Y757.}}
\author{Jose Divas\'on \and
  Sebastiaan Joosten \and
  Ren\'e Thiemann \and
  Akihisa Yamada}
\maketitle

\begin{abstract}
We formalize the Berlekamp-Zassenhaus algorithm for factoring 
square-free integer polynomials in Isabelle/HOL.
We further adapt an existing formalization of
Yun's square-free factorization algorithm
to integer polynomials, and thus
provide an efficient and certified factorization
algorithm for arbitrary univariate polynomials.

The algorithm first performs a factorization in the prime field GF($p$)
and then performs computations in the integer ring modulo $p^k$,
where both $p$ and $k$ are determined at runtime.
Since a natural modeling of these structures via dependent types is
not possible in Isabelle/HOL,
we formalize the whole algorithm using Isabelle's recent 
addition of local type definitions. 

Through experiments we verify that
our algorithm factors polynomials of degree 100 within seconds.
\end{abstract}

\tableofcontents

\section{Introduction}

Modern algorithms to factor integer polynomials
-- following Berlekamp and Zassenhaus --
work via polynomial factorization over prime fields $\GFp$ and quotient rings \ring
\cite{Berlekamp,CZ81}.
Algorithm~\ref{bz} illustrates the basic structure of such an algorithm.\footnote{Our 
algorithm starts with step \ref{p:prime}, so that
section numbers and step-numbers coincide.}

\begin{algorithm}[h]
\caption{A modern factorization algorithm\label{bz}}
\setcounter{AlgoLine}{3} % start at the number after this line
\KwIn{Square-free integer polynomial $f$.}
\KwOut{Irreducible factors $f_1,\dots,f_n$ such that
$f = f_1 \cdot \ldots \cdot f_n$.}
%
\lnl{p:prime} Choose a suitable prime $p$ depending on $f$.\;
\lnl{p:berlekamp}
  Factor $f$ in \GFp: $f \equiv g_1 \cdot\ldots\cdot g_m \pmod p$.\;
\lnl{p:exp} Determine a suitable bound $d$ on the degree, depending on $g_1,\ldots,g_m$.
  Choose an exponent $k$ such that every coefficient of a factor of a given multiple of $f$ in $\ints$ 
  with degree at most $d$ can be uniquely represent by a number below $p^k$. \;
\lnl{p:hensel}
 From step \ref{p:berlekamp} compute the
  unique factorization $f \equiv h_1 \cdot \ldots \cdot h_m \pmod {p^k}$ via the Hensel lifting.\;
\lnl{p:integer}
 Construct a factorization $f = f_1 \cdot \ldots \cdot f_n$ 
  over the integers where each $f_i$ corresponds to the product of one or more $h_j$.
\end{algorithm}

In previous work on algebraic numbers \cite{TY16}, we implemented
Algorithm~\ref{bz} in Isabelle/HOL \cite{Isabelle} as a 
function of type $\tint\ \tpoly \tto \tint\ \tpoly\ \tlist$,
where we chose Berlekamp's algorithm in step \ref{p:berlekamp}.
However, the algorithm was available only as an oracle,
and thus a validity check on the result factorization had to be performed.

In this work we fully formalize the correctness of our implementation.

\begin{theorem}[Berlekamp-Zassenhaus' Algorithm]
\label{thm:bz}
\begin{align*}
& \assumes\ \sqfree\ (f :: \tint\ \tpoly) \\
& \quad\iand\ \idegree\ f \neq 0 \\
& \quad\iand\ \bz\ f = \fs \\
& \shows\ f = \listprod\ \fs\ \\
& \quad\iand\ \forall f_i \in \set\ \fs.\ \irred\ f_i
\end{align*}
\end{theorem}

%
% 
%now provide full proofs  changes the previous implementation  correctness of the implementation is not yet formalized in Isabelle/HOL.
%Hence it is invoked in a certified wrapper which takes
%an arbitrary integer polynomial as input, performs the desired preprocessing,
%i.e., square-free and content-free factorization, and passes each
%preliminary factor $f$ to $\oracle$. 
%It finally tests the validity of the obtained factorizations
%$f = f_1 \cdot \ldots \cdot f_n$, but it does not test optimality, i.e.,
%irreducibility of the resulting factors.   
%
%The current work is a significant step forward to formally proving the\linebreak soundness 
%of $\oracle$, namely by formally proving the soundness of Berlekamp's algorithm
%in step~\ref{p:berlekamp}. 

To obtain \rTH{thm:bz} we perform the following tasks.

\begin{itemize}
\item
  We introduce two formulations of $\GFp$ and $\ring$.
  We first define a 
  type to represent these domains, 
  employing ideas from HOL multivariate analysis. 
  This is essential
  for reusing many type-based algorithms from the Isabelle distribution
  and the AFP (archive of formal proofs).
  At some points in our developement,
  the type-based setting is still too restrictive.
  Hence we also introduce a second formulation which is \emph{locale-based}.

\item The prime $p$ in step \ref{p:prime} must be chosen so that $f$ remains square-free
  in $\GFp$.
  For the termination of the algorithm, we prove that such a prime always
  exists. 

\item
  We explain Berlekamp's algorithm that factors polynomials over prime fields,
  and formalize its correctness using the type-based representation.
  Since Isabelle's code generation does not work for the type-based representation of prime fields,
  we define an implementation of Berlekamp's algorithm which avoids
  type-based polynomial algorithms and type-based prime fields.
  The soundness of this implementation is proved via the transfer package \cite{lifting}:
  we transform the type-based soundness statement of Berlekamp's algorithm
  into a statement which speaks solely about integer polynomials.
  Here, we crucially rely upon local type definitions 
  \cite{KP16} to eliminate the presence of the type for the prime field $\GFp$.
    
\item For step \ref{p:exp} we need to find a bound on the coefficients of 
  the factors of a polynomial.
  For this purpose, we formalize Mignotte's factor bound.
  During this formalization task
  we detected a bug in our previous oracle implementation,
  which computed improper bounds on the degrees of factors.
  
\item We formalize the Hensel lifting.
  As for Berlekamp's algorithm,
  we first formalize basic operations in the type-based setting.
  Unfortunately, however, this result cannot be extended to the full Hensel lifting.
  Therefore, we model the Hensel lifting in a locale-based way so that
  modulo operation is explicitly applied on polynomials.
  
\item For the reconstruction in step \ref{p:integer} we closely
  follow the description of Knuth \cite[page~452]{Knuth}. Here, we use the same
  representation of polynomials over $\ring$ as for the Hensel lifting.
  

\item We adapt an existing square-free factorization algorithm from
  $\rats$ to $\ints$. In combination with the previous results this leads to a factorization algorithm
  for arbitrary integer and rational polynomials. 

%\item Moreover, we formalize (efficient) division algorithms for non-field polynomials
%  that are applied within the oracle,
%  and also optimize the existing division algorithm for field polynomials (\rSC{polydiv}).
%  The improvements are now integrated in the Isabelle distribution as code equations
%  \cite{DataRefinement,codegen}.
%  
%\item A comparison of the trusted code with the one from $\oracle$ revealed two
%  mistakes which are now repaired (\rSC{compare oracle}).
%
%\item Mignotte-bound (somewhere)
%\item Hensel-lifting (somewhere)
%\item Reconstruction (somewhere)
\end{itemize}

%Related work:
To our knowledge, this is the first formalization of the Berlekamp-Zassenhaus algorithm.
For instance, Barthe \etal report that there is no formalization of an efficient factorization algorithm over $\GFp$ available in Coq \cite[Section 6, note 3 on formalization]{NoCoqFactorization}.

Some key theorems leading to the algorithm have already been
formalized in Isabelle or other proof assistants.
In ACL2, for instance, polynomials over a field are shown to be a unique factorization domain
(UFD)~\cite{cowles2006unique}.
A more general result,
namely that polynomials over UFD are also UFD,
was already developed in Isabelle/HOL for implementing algebraic numbers \cite{TY16}
and an independent development by Eberl is now available in the Isabelle distribution.

An Isabelle formalization of Hensel's lemma is provided by Kobayashi \etal \cite{Kobayashi2005},
who defined the valuations of polynomials via Cauchy sequences, and used this setup to prove the lemma.
Consequently, their result requires a `valuation ring' as precondition in their formalization.
While this extra precondition is theoretically met in our setting,
we did not attempt to reuse their results,
because the type of polynomials in their formalization (from HOL-Algebra) differs 
from the polynomials in our development (from HOL/Library).
Instead, we formalize a direct proof for Hensel's lemma.
Our formalizations are incomparable:
On the one hand, Kobayashi \etal did not consider only integer polynomials as we do.
On the other hand, we additionally formalize the quadratic Hensel lifting~\cite{Zassenhaus69},
extend the lifting from binary to $n$-ary factorizations,
and prove a uniqueness result,
which is required for proving the soundness of \rTH{thm:bz}.

A Coq formalization of Hensel's lemma is also available, %~\cite{Martin-Dorel:2011aa},
which is used for certifying integral roots and `hardest-to-round computation'~\cite{Martin-Dorel2015}.
If one is interested in certifying a factorization,
rather than a certified algorithm that performs it,
it suffices to test that all the found factors are irreducible.
Kirkels \cite{kirkels2004} formalized a sufficient criterion for this test in Coq:
when a polynomial is irreducible modulo some prime, it is also irreducible in $\mathbb{Z}$.
Both formalizations are in Coq, and we did not attempt to reuse them.

% include generated text of all theories
\input{session}



\bibliographystyle{abbrv}
\bibliography{root}

\end{document}
