\documentclass[11pt,a4paper]{article}
\usepackage{isabelle,isabellesym}
\usepackage{amssymb,amsmath,amsthm}

% this should be the last package used
\usepackage{pdfsetup}


\theoremstyle{definition}
\newtheorem{theorem}{Theorem}
% urls in roman style, theory text in math-similar italics
\urlstyle{rm}
\isabellestyle{it}

\newcommand{\OSC}{\mathit{OSC}}
\newcommand{\abs}[1]{\left\lvert #1 \right\rvert}

\begin{document}

\title{Maximum Cardinality Matching}
\author{Christine Rizkallah}
\maketitle

\begin{abstract}

A \emph{matching} in a graph $G$ is a subset $M$ of the edges of $G$ such that 
no two share an endpoint.  
A matching has maximum cardinality if its cardinality is at least as large as
that of any other matching.  
An \emph{odd-set cover} $\OSC$ of a graph $G$ is a labeling of the nodes of $G$
with integers such that every edge of $G$ is either incident to a node labeled 1 or connects two nodes labeled with the same number $i \ge 2$.

\begin{theorem}[Edmonds~\cite{Edmonds:matching}]

\label{thm-edm}
Let $M$ be a matching in a graph $G$ and let $OSC$ be an odd-set cover of $G$.
For any $i \ge 0$, let $n_i$ be the number of nodes labeled $i$. If 
$$\abs{M} = n_1 + \sum_{i\ge 2}\lfloor n_i/2 \rfloor$$
then $M$ is a maximum cardinality matching.
\end{theorem}

We provide an Isabelle proof of Edmonds theorem. For an explanation of the proof see \cite{VerificationofCertifyingComputations}.
%\begin{proof} Let $N$ be any matching in $G$. 
%For $i \ge 2$, let $N_i$ be the edges in $N$ that connect two nodes labeled $i$
%and let $N_1$ be the remaining edges in $N$. Then, by the definition of odd-set
%cover, every edge in $N_1$ is incident to a vertex labeled 1. Since edges in a
%matching do not share endpoints, we have
%\[\abs{N_1} \le n_1\;\;\text{and}\;\;\abs{N_i} \le \lfloor n_i/2 \rfloor \;\;\text{for
%$i \ge 2$.}\]
%Thus $\abs{N} \le  n_1 + \sum_{i\ge 2} \lfloor n_i / 2 \rfloor =
%\abs{M}$. 
%\qed
%\end{proof}
%In the following we present an  Isabelle/HOL proof of Theorem \ref{thm-edm}.
%The Isabelle/HOL proof follows the scheme of the informal proof and is split
%into two main parts. 
%
%For $i \ge 2$, let $M_i$ be the edges in $M$ that connect two nodes labeled $i$
%and let $M_1$ be the remaining edges in $M$. We use the definition of odd-set
%cover to prove that $M \subseteq  \bigcup_{i\ge1} M_i$ and thus $\abs{M} \le \sum_{i}\abs{M_i}$. 
%Let $V_i$ be the nodes labeled $i$ and let $n_i = \abs{V_i}$. 
%We formally prove: $\abs{M_1} \le n_1$ and $\abs{M_i} \le \lfloor n_i/2
%\rfloor$. 
%
%\newcommand{\pinV}{\mathit{endpoint}_{V_1}}
%
%In order to prove $\abs{M_1} \le n_1$, we exhibit an injective function from $M_1$
%to $V_1$. We first prove, using the definition of odd-set cover, that every
%edge $e \in M_1$ has at least one endpoint in $V_1$. This gives rise to a function
%$\pinV: M_1 \mapsto V_1$. We then use the fact that edges in
%a matching do not share endpoints, i.e., are disjoint when interpreted as
%sets, to conclude that $\pinV$ is injective. This establishes $\abs{M_1} \le
%\abs{V_i}$. 
%
%For $i\ge2$ the proof of the inequality $\abs{M_i} \le \lfloor n_i/2 \rfloor$ is similar,
%but more involved. $M_i$ is a set of edges.
%If we represent edges as sets (each has cardinality equals two), then $M_i$ is a collection of sets.
%We define the set of vertices $V^\prime_i$ to be $\bigcup M_i$ and use the definition of odd-set cover to
%prove $V^\prime_i \subseteq V_i$. 
%Then, we use the fact that the edges in a matching are pairwise disjoint to 
% prove $\abs{V^\prime_i} = 2 * \abs{M_i}$. 
%Note also that $\abs{V^\prime_i}$ must be even since $\abs{M_i}$ is a natural
%number. Thus we can prove that $\abs{M_i} \le \lfloor\abs{V^\prime_i}/2\rfloor$ and hence 
%$\abs{M_i} \le \lfloor\abs{V^\prime_i} / 2\rfloor \le \lfloor \abs{V_i}/2 \rfloor
%= \lfloor n_i/2 \rfloor$. 

\end{abstract}
\tableofcontents

% sane default for proof documents
\parindent 0pt\parskip 0.5ex

% generated text of all theories
\input{session}

% optional bibliography
\bibliographystyle{abbrv}
\bibliography{root}

\end{document}

%%% Local Variables:
%%% mode: latex
%%% TeX-master: t
%%% End:
