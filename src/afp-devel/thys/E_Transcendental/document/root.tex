\documentclass[11pt,a4paper]{article}
\usepackage[T1]{fontenc}
\usepackage{isabelle,isabellesym}

% further packages required for unusual symbols (see also
% isabellesym.sty), use only when needed

\usepackage{amssymb, amsmath}
  %for \<leadsto>, \<box>, \<diamond>, \<sqsupset>, \<mho>, \<Join>,
  %\<lhd>, \<lesssim>, \<greatersim>, \<lessapprox>, \<greaterapprox>,
  %\<triangleq>, \<yen>, \<lozenge>

%\usepackage{eurosym}
  %for \<euro>

%\usepackage[only,bigsqcap]{stmaryrd}
  %for \<Sqinter>

%\usepackage{eufrak}
  %for \<AA> ... \<ZZ>, \<aa> ... \<zz> (also included in amssymb)

%\usepackage{textcomp}
  %for \<onequarter>, \<onehalf>, \<threequarters>, \<degree>, \<cent>,
  %\<currency>

% this should be the last package used
\usepackage{pdfsetup}

% urls in roman style, theory text in math-similar italics
\urlstyle{rm}
\isabellestyle{it}

% for uniform font size
%\renewcommand{\isastyle}{\isastyleminor}


\begin{document}

\title{The Transcendence of $e$}
\author{Manuel Eberl}
\maketitle

\begin{abstract}
This work contains a formalisation of the proof that Euler's number $e$ is transcendental. The proof follows the standard approach of assuming that $e$ is algebraic and then using a specific integer polynomial to derive two inconsistent bounds, leading to a contradiction.

This approach can be found in many different sources; this formalisation mostly follows a PlanetMath article~\cite{planetmath} by Roger Lipsett.
\end{abstract}

\tableofcontents

\parindent 0pt\parskip 0.5ex

\input{session}

\bibliographystyle{abbrv}
\begingroup
\raggedright
\bibliography{root}
\endgroup

\end{document}

%%% Local Variables:
%%% mode: latex
%%% TeX-master: t
%%% End:
