\documentclass[11pt,a4paper]{article}
\usepackage[T1]{fontenc}
\usepackage{isabelle,isabellesym}
\usepackage{amsfonts, amsmath, amssymb}

%\usepackage{eurosym}
  %for \<euro>

%\usepackage[only,bigsqcap,bigparallel,fatsemi,interleave,sslash]{stmaryrd}
  %for \<Sqinter>, \<Parallel>, \<Zsemi>, \<Parallel>, \<sslash>

%\usepackage{eufrak}
  %for \<AA> ... \<ZZ>, \<aa> ... \<zz> (also included in amssymb)

%\usepackage{textcomp}
  %for \<onequarter>, \<onehalf>, \<threequarters>, \<degree>, \<cent>,
  %\<currency>

% this should be the last package used
\usepackage{pdfsetup}

% urls in roman style, theory text in math-similar italics
\urlstyle{rm}
\isabellestyle{it}

% for uniform font size
%\renewcommand{\isastyle}{\isastyleminor}


\begin{document}

\title{Actuarial Mathematics}
\author{Yosuke Ito}
\maketitle

\begin{abstract}
  Actuarial Mathematics is a theory in applied mathematics,
  which is mainly used for determining the prices of insurance products
  and evaluating the liability of a company associating with insurance contracts.
  It is related to calculus, probability theory and financial theory, etc.

  In this entry, I formalize the very basic part of Actuarial Mathematics in Isabelle/HOL.
  The first formalization is about the theory of interest
  which deals with interest rates, present value factors, an annuity certain, etc.

  I have already formalized the basic part of Actuarial Mathematics in Coq
  (https://github.com/Yosuke-Ito-345/Actuary).
  This entry is currently the partial translation and
  a little generalization of the Coq formalization.
  The further translation in Isabelle/HOL is now proceeding.
\end{abstract}

\tableofcontents

% sane default for proof documents
\parindent 0pt\parskip 0.5ex

% generated text of all theories
\input{session}

% optional bibliography
%\bibliographystyle{abbrv}
%\bibliography{root}

\end{document}

%%% Local Variables:
%%% mode: latex
%%% TeX-master: t
%%% End:
