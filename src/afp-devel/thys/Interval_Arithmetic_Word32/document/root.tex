\documentclass[11pt,a4paper]{article}
\usepackage{isabelle,isabellesym}

% further packages required for unusual symbols (see also
% isabellesym.sty), use only when needed

\usepackage{amssymb}
  %for \<leadsto>, \<box>, \<diamond>, \<sqsupset>, \<mho>, \<Join>,
  %\<lhd>, \<lesssim>, \<greatersim>, \<lessapprox>, \<greaterapprox>,
  %\<triangleq>, \<yen>, \<lozenge>

%\usepackage{eurosym}
  %for \<euro>

\usepackage[only,bigsqcap]{stmaryrd}
  %for \<Sqinter>

%\usepackage{eufrak}
  %for \<AA> ... \<ZZ>, \<aa> ... \<zz> (also included in amssymb)

%\usepackage{textcomp}
  %for \<onequarter>, \<onehalf>, \<threequarters>, \<degree>, \<cent>,
  %\<currency>

% this should be the last package used
\usepackage{pdfsetup}

% urls in roman style, theory text in math-similar italics
\urlstyle{rm}
\isabellestyle{it}

% for uniform font size
%\renewcommand{\isastyle}{\isastyleminor}


\begin{document}

\title{Interval Arithmetic on 32-bit Words}
\author{Brandon Bohrer}
\maketitle

\begin{abstract}
 This article implements conservative interval arithmetic
 computations, then uses this interval arithmetic to implement a
 simple programming language where all terms have 32-bit signed word
 values, with explicit infinities for terms outside the representable
 bounds. Our target use case is interpreters for languages that must
 have a well-understood low-level behavior.

 We include a formalization of bounded-length strings which are used
 for the identifiers of our language. Bounded-length identifiers are
 useful in some applications, for example the
 Differential\_Dynamic\_Logic \cite{Differential_Dynamic_Logic-AFP}
 article, where a Euclidean space indexed by identifiers demands that
 identifiers are finitely many.
\end{abstract}
\newpage

\tableofcontents
\newpage

% sane default for proof documents
\parindent 0pt\parskip 0.5ex

% generated text of all theories
\input{session}

\bibliographystyle{abbrv}
\bibliography{root}

\end{document}

%%% Local Variables:
%%% mode: latex
%%% TeX-master: t
%%% End:
