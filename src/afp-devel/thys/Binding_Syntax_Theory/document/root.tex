\documentclass[11pt,a4paper]{article}
\usepackage[T1]{fontenc}
\usepackage{isabelle,isabellesym}

% further packages required for unusual symbols (see also
% isabellesym.sty), use only when needed

%\usepackage{amssymb}
  %for \<leadsto>, \<box>, \<diamond>, \<sqsupset>, \<mho>, \<Join>,
  %\<lhd>, \<lesssim>, \<greatersim>, \<lessapprox>, \<greaterapprox>,
  %\<triangleq>, \<yen>, \<lozenge>

%\usepackage{eurosym}
  %for \<euro>

%\usepackage[only,bigsqcap]{stmaryrd}
  %for \<Sqinter>

%\usepackage{eufrak}
  %for \<AA> ... \<ZZ>, \<aa> ... \<zz> (also included in amssymb)

%\usepackage{textcomp}
  %for \<onequarter>, \<onehalf>, \<threequarters>, \<degree>, \<cent>,
  %\<currency>

% this should be the last package used
\usepackage{pdfsetup}

% urls in roman style, theory text in math-similar italics
\urlstyle{rm}
\isabellestyle{it}

% for uniform font size
%\renewcommand{\isastyle}{\isastyleminor}


\begin{document}

\title{A General Theory of Syntax with Bindings}
\author{Lorenzo Gheri and Andrei Popescu}
\maketitle

\begin{abstract}
We formalize a theory of syntax with bindings that has been developed and refined over the last decade to support several large formalization efforts. Terms are defined for an arbitrary number of constructors of varying numbers of inputs, quotiented to alpha-equivalence and sorted according to a binding signature. The theory includes many properties of the standard operators on terms: substitution, swapping and freshness. It also includes bindings-aware induction and recursion principles and support for semantic interpretation. 
This work has been presented in the ITP 2017 paper ``A Formalized General Theory of Syntax with Bindings''. 
\end{abstract}

\tableofcontents

% sane default for proof documents
\parindent 0pt\parskip 0.5ex

% generated text of all theories
\input{session}

% optional bibliography
%\bibliographystyle{abbrv}
%\bibliography{root}

\end{document}

%%% Local Variables:
%%% mode: latex
%%% TeX-master: t
%%% End:
