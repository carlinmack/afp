\documentclass[11pt,a4paper]{article}

\usepackage{isabelle,isabellesym}
\usepackage{amssymb,ragged2e}
\usepackage{pdfsetup}

\isabellestyle{it}
\renewenvironment{isamarkuptext}{\par\isastyletext\begin{isapar}\justifying\color{blue}}{\end{isapar}}
\renewcommand\labelitemi{$*$}

\begin{document}

\title{Relational Characterisations of Paths}
\author{Walter Guttmann and Peter H\"ofner}
\maketitle

\begin{abstract}
  Binary relations are one of the standard ways to encode, characterise and reason about graphs.
  Relation algebras provide equational axioms for a large fragment of the calculus of binary relations.
  Although relations are standard tools in many areas of mathematics and computing, researchers usually fall back to point-wise reasoning when it comes to arguments about paths in a graph.
  We present a purely algebraic way to specify different kinds of paths in Kleene relation algebras, which are relation algebras equipped with an operation for reflexive transitive closure.
  We study the relationship between paths with a designated root vertex and paths without such a vertex.
  Since we stay in first-order logic this development helps with mechanising proofs.
  To demonstrate the applicability of the algebraic framework we verify the correctness of three basic graph algorithms.
\end{abstract}

\tableofcontents

\section*{Overview}

A path in a graph can be defined as a connected subgraph of edges where each vertex has at most one incoming edge and at most one outgoing edge \cite{Diestel2005,Tinhofer1976}.
We develop a theory of paths based on this representation and use it for algorithm verification.
All reasoning is done in variants of relation algebras and Kleene algebras \cite{Kozen1994,Ng1984,Tarski1941}.

Section 1 presents fundamental results that hold in relation algebras.
Relation-algebraic characterisations of various kinds of paths are introduced and compared in Section 2.
We extend this to paths with a designated root in Section 3.
Section 4 verifies the correctness of a few basic graph algorithms.

These Isabelle/HOL theories formally verify results in \cite{BerghammerFurusawaGuttmannHoefner2020}.
See this paper for further details and related work.

\begin{flushleft}
\input{session}
\end{flushleft}

\bibliographystyle{abbrv}
\bibliography{root}

\end{document}

