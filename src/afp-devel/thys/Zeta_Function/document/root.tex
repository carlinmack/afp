\documentclass[11pt,a4paper]{article}
\usepackage{isabelle,isabellesym}
\usepackage{amsfonts, amsmath, amssymb}

% this should be the last package used
\usepackage{pdfsetup}

% urls in roman style, theory text in math-similar italics
\urlstyle{rm}
\isabellestyle{it}


\begin{document}

\title{The Hurwitz and Riemann $\zeta$ functions}
\author{Manuel Eberl}
\maketitle

\begin{abstract}
This entry builds upon the results about formal and analytic Dirichlet series to define the Hurwitz $\zeta$ function $\zeta(a,s)$ and,
based on that, the Riemann $\zeta$ function $\zeta(s)$. This is done by first defining them for $\mathfrak{R}(z) > 1$ and then successively
extending the domain to the left using the Euler--MacLaurin formula.

Apart from the most basic facts such as analyticity, the following results are provided:
\begin{itemize}
\item the Stieltjes constants and the Laurent expansion of $\zeta(s)$ at $s = 1$
\item the non-vanishing of $\zeta(s)$ for $\mathfrak{R}(s)\geq 1$
\item the relationship between $\zeta(a,s)$ and $\Gamma$
\item the special values at negative integers and positive even integers
\item Hurwitz's formula and the reflection formula for $\zeta(s)$
\item the Hadjicostas--Chapman formula~\cite{chapman2004,hadjicostas2004}
\end{itemize}

The entry also contains Euler's analytic proof of the infinitude of primes, based on the fact that $\zeta(s)$ has a pole at $s = 1$.
\end{abstract}

\newpage
\tableofcontents
\newpage
\parindent 0pt\parskip 0.5ex

\input{session}

\bibliographystyle{abbrv}
\bibliography{root}

\end{document}

%%% Local Variables:
%%% mode: latex
%%% TeX-master: t
%%% End:
