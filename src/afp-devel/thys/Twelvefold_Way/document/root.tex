\documentclass[11pt,a4paper]{article}
\usepackage{isabelle,isabellesym}

% this should be the last package used
\usepackage{pdfsetup}

% urls in roman style, theory text in math-similar italics
\urlstyle{rm}
\isabellestyle{it}

\begin{document}

\title{The Twelvefold Way}
\author{Lukas Bulwahn}
\maketitle

\begin{abstract}

This entry provides all cardinality theorems of the Twelvefold Way.
The Twelvefold Way~\cite{bogart-2004, stanley-2012, wikipedia:Twelvefold-Way}
systematically classifies twelve related combinatorial problems
concerning two finite sets, which include
counting permutations, combinations, multisets, set partitions and number partitions.
This development builds upon the existing formal
developments~\cite{Card_Partitions-AFP, Card_Multisets-AFP, Card_Number_Partitions-AFP}
with cardinality theorems for those structures.
It provides twelve bijections from the various structures to
different equivalence classes on finite functions, and hence, proves
cardinality formulae for these equivalence classes on finite functions.
\end{abstract}

\tableofcontents

% sane default for proof documents
\parindent 0pt\parskip 0.5ex

% generated text of all theories
\input{session}

\bibliographystyle{abbrv}
\bibliography{root}

\end{document}

%%% Local Variables:
%%% mode: latex
%%% TeX-master: t
%%% End:
