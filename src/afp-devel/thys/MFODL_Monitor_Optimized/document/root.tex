\documentclass[10pt,a4paper]{article}
\usepackage{isabelle,isabellesym}

\usepackage{a4wide}
\usepackage[english]{babel}
\usepackage{eufrak}
\usepackage{amssymb}

% this should be the last package used
\usepackage{pdfsetup}

% urls in roman style, theory text in math-similar italics
\urlstyle{rm}
\isabellestyle{literal}


\begin{document}

\title{Formalization of an Optimized Monitoring Algorithm for\\ Metric First-Order Dynamic Logic with Aggregations}
\author{Thibault Dardinier \and Lukas Heimes \and Martin Raszyk \and Joshua Schneider \and Dmitriy Traytel}

\maketitle

\begin{abstract}
A monitor is a runtime verification tool that solves the following problem: Given a stream
of time-stamped events and a policy formulated in a specification language, decide
whether the policy is satisfied at every point in the stream. We verify the correctness
of an executable monitor for specifications given as formulas in metric first-order
dynamic logic (MFODL), which combines the features of metric first-order temporal logic
(MFOTL)~\cite{BasinKMZ-JACM15} and metric dynamic logic~\cite{BasinKT-RV17}. Thus, MFODL
supports real-time constraints, first-order parameters, and regular expressions.
Additionally, the monitor supports aggregation operations such as count and sum. This
formalization, which is described in a paper at IJCAR
2020~\cite{BasinDHKRST2020IJCAR}, significantly extends
\href{https://www.isa-afp.org/entries/MFOTL_Monitor.html}{previous work on a verified
monitor} for MFOTL~\cite{SchneiderBKT2019RV}. Apart from the addition of regular
expressions and aggregations, we implemented
\href{https://www.isa-afp.org/entries/Generic_Join.html}{multi-way joins} and
a specialized sliding window algorithm to further optimize the monitor.
\end{abstract}

\tableofcontents

% sane default for proof documents
\parindent 0pt\parskip 0.5ex

% generated text of all theories
\input{session}

\bibliographystyle{abbrv}
\bibliography{root}

\end{document}

%%% Local Variables:
%%% mode: latex
%%% TeX-master: t
%%% End:
