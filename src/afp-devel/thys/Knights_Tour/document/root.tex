\documentclass[11pt,a4paper]{article}
\usepackage[T1]{fontenc}
\usepackage{isabelle,isabellesym}

% further packages required for unusual symbols (see also
% isabellesym.sty), use only when needed

%\usepackage{amssymb}
  %for \<leadsto>, \<box>, \<diamond>, \<sqsupset>, \<mho>, \<Join>,
  %\<lhd>, \<lesssim>, \<greatersim>, \<lessapprox>, \<greaterapprox>,
  %\<triangleq>, \<yen>, \<lozenge>

%\usepackage{eurosym}
  %for \<euro>

%\usepackage[only,bigsqcap,bigparallel,fatsemi,interleave,sslash]{stmaryrd}
  %for \<Sqinter>, \<Parallel>, \<Zsemi>, \<Parallel>, \<sslash>

%\usepackage{eufrak}
  %for \<AA> ... \<ZZ>, \<aa> ... \<zz> (also included in amssymb)

%\usepackage{textcomp}
  %for \<onequarter>, \<onehalf>, \<threequarters>, \<degree>, \<cent>,
  %\<currency>

\usepackage{amsmath}
\usepackage{float}

% this should be the last package used
\usepackage{pdfsetup}

% urls in roman style, theory text in math-similar italics
\urlstyle{rm}
\isabellestyle{it}

% for uniform font size
%\renewcommand{\isastyle}{\isastyleminor}

\begin{document}

\title{Knight's Tour Revisited Revisited}
\author{Lukas Koller\\Department of Informatics\\Technical University of Munich}
\maketitle

\begin{abstract}
This is a formalization of the article ``Knight's Tour Revisited'' by
Cull and De Curtins where they prove the existence of a Knight's path for arbitrary $n\times m$-boards with 
$\operatorname{min}(n,m) \geq 5$. If $n\cdot m$ is even, then there exists a Knight's circuit.

A Knight's Path is a sequence of moves of a Knight on a chessboard s.t. the Knight visits every square of a chessboard exactly once.
Finding a Knight's path is a an instance of the Hamiltonian path problem.

During the formalization two mistakes in the original proof were discovered. 
These mistakes are corrected in this formalization.
\end{abstract}

\tableofcontents

% sane default for proof documents
\parindent 0pt\parskip 0.5ex

% generated text of all theories
\input{session}

% optional bibliography
\bibliographystyle{abbrv}
\bibliography{root}

\end{document}

%%% Local Variables:
%%% mode: latex
%%% TeX-master: t
%%% End:
