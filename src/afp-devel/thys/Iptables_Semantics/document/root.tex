\documentclass[11pt,a4paper]{article}
\usepackage[T1]{fontenc}
\usepackage{isabelle,isabellesym}

% further packages required for unusual symbols (see also
% isabellesym.sty), use only when needed

\usepackage{amssymb}
  %for \<leadsto>, \<box>, \<diamond>, \<sqsupset>, \<mho>, \<Join>,
  %\<lhd>, \<lesssim>, \<greatersim>, \<lessapprox>, \<greaterapprox>,
  %\<triangleq>, \<yen>, \<lozenge>


%semantic rules printing
\usepackage{mathpartir}


% this should be the last package used
\usepackage{pdfsetup}



% urls in roman style, theory text in math-similar italics
\urlstyle{rm}
\isabellestyle{it}

% for uniform font size
%\renewcommand{\isastyle}{\isastyleminor}


\begin{document}

\title{Iptables-Semantics}
\author{Cornelius Diekmann, Lars Hupel}
\maketitle

\begin{abstract}  
  We present a big step semantics of the filtering behavior of the Linux/netfilter iptables firewall. 
  We provide algorithms to simplify complex iptables rulests to a simple firewall model (c.f.\ AFP entry Simple\_Firewall) and to verify spoofing protection of a ruleset. 
  Internally, we embed our semantics into ternary logic, ultimately supporting every iptables match condition by abstracting over unknowns. 
  Using this AFP entry and all entries it depends on, we created an easy-to-use, stand-alone haskell tool called \emph{fffuu} (\url{http://iptables.isabelle.systems}). 
  The tool does not require any input ---except for the \texttt{iptables-save} dump of the analyzed firewall--- and presents interesting results about the user's ruleset. 
  Real-Word firewall errors have been uncovered, as well as the correctness of rulesets has been proven with the help of our tool. 
  
For a detailed description, see \cite{diekmann2015fm,diekmann2015cnsm,diekmann2016networking,diekmann2015congress}. 
\end{abstract}

\paragraph*{Acknowledgements}
This entry would not have been possible without the help of Julius Michaelis, Max Haslbeck, Stephan-A.\ Posselt, Lars Noschinski, Manuel Eberl, Gerwin Klein, the Isabelle group Munich, and Georg Carle.
\bigskip

\tableofcontents

% sane default for proof documents
\parindent 0pt\parskip 0.5ex

% generated text of all theories
\input{session}

% optional bibliography
\bibliographystyle{abbrv}
\bibliography{root}

\end{document}

%%% Local Variables:
%%% mode: latex
%%% TeX-master: t
%%% End:
