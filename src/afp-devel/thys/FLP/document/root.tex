\documentclass[11pt,a4paper]{article}
\usepackage[english]{babel}
\usepackage[utf8]{inputenc}
\usepackage{csquotes}
\usepackage{amsmath}
\usepackage{isabelle,isabellesym}

\usepackage{color}

\usepackage{custom-macros}

\usepackage[top=3cm,bottom=4.5cm]{geometry}

% this should be the last package used
\usepackage{pdfsetup}

% urls in roman style, theory text in isabelle-similar-similar type-writer
\urlstyle{rm}
\isabellestyle{tt}

\title{A constructive proof for FLP}
\author{Benjamin Bisping \and Paul-David Brodmann \and Tim Jungnickel \and
  Christina Rickmann \and Henning Seidler \and Anke St\"uber  \and Arno Wilhelm-Weidner \and
 Kirstin Peters \and Uwe Nestmann}
%
\date{\today}

\begin{document}

\maketitle

\begin{abstract}
	The impossibility of distributed consensus with one faulty process is a result with important consequences for real world distributed systems e.g., commits in replicated databases. 
	Since proofs are not immune to faults and even plausible proofs with a profound formalism can conclude wrong results, we validate the fundamental result named FLP after Fischer, Lynch and Paterson by using the interactive theorem prover Isabelle/HOL.
	We present a formalization of distributed systems and the aforementioned consensus problem.
	Our proof is based on Hagen Völzer's paper \emph{A constructive proof for FLP}.
	In addition to the enhanced confidence in the validity of Völzer's proof, we contribute the missing gaps to show the correctness in Isabelle/HOL. 
	We clarify the proof details and even prove fairness of the infinite execution that contradicts consensus. 
	Our Isabelle formalization can also be reused for further proofs of properties of distributed systems.
\end{abstract}
In the following we present the Isabelle implementation of the underlying theory as well as all proofs of the results presented in the paper \emph{Mechanical Verification of a Constructive Proof for FLP} as submitted to the Proceedings of the \textit{seventh conference on Interactive Theorem Proving}, ITP 2016, LNCS. 
\newpage

\tableofcontents
\newpage

% include generated text of all theories
\input{session}

\bibliographystyle{abbrv}
\bibliography{root}

\end{document}
