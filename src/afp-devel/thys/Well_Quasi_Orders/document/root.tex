\documentclass[11pt,a4paper]{article}
\usepackage{isabelle,isabellesym}

% this should be the last package used
\usepackage{pdfsetup}

% urls in roman style, theory text in math-similar italics
\urlstyle{rm}
\isabellestyle{it}


\begin{document}

\title{Well-Quasi-Orders}
\author{Christian Sternagel\thanks{%
  The research was funded by the Austrian Science Fund (FWF): J3202.}}
\maketitle

\begin{abstract}
Based on Isabelle/HOL's type class for preorders, we introduce a type class for
well-quasi-orders (wqo) which is characterized by the absence of ``bad''
sequences (our proofs are along the lines of the proof of Nash-Williams
\cite{N1963}, from which we also borrow terminology).  Our main results are
instantiations for the product type, the list type, and a type of finite trees,
which (almost) directly follow from our proofs of (1) Dickson's Lemma, (2)
Higman's Lemma, and (3) Kruskal's Tree Theorem. More concretely:
\begin{enumerate}
\item If the sets $A$ and $B$ are wqo then their Cartesian product is wqo.
\item If the set $A$ is wqo then the set of finite lists over $A$ is wqo.
\item If the set $A$ is wqo then the set of finite trees over $A$ is wqo.
\end{enumerate}
\end{abstract}

\tableofcontents


% sane default for proof documents
\parindent 0pt\parskip 0.5ex

% generated text of all theories
\input{session}

% optional bibliography
\bibliographystyle{abbrv}
\bibliography{root}

\end{document}

%%% Local Variables:
%%% mode: latex
%%% TeX-master: t
%%% End:
