\documentclass[11pt,a4paper]{article}
\usepackage{isabelle,isabellesym}
\usepackage{mathtools}
\usepackage{amssymb}

% this should be the last package used
\usepackage{pdfsetup}

% urls in roman style, theory text in math-similar italics
\urlstyle{rm}
\isabellestyle{it}

\DeclarePairedDelimiter{\norm}{\lVert}{\rVert}

\begin{document}

\title{Gromov hyperbolic spaces in Isabelle}
\author{Sebastien Gouezel}
\date{}
\maketitle

\begin{abstract}
A geodesic metric space is Gromov hyperbolic if all its geodesic triangles
are thin, i.e., every side is contained in a fixed thickening of the two
other sides. While this definition looks innocuous, it has proved extremely
important and versatile in modern geometry since its introduction by
Gromov. We formalize the basic classical properties of Gromov hyperbolic
spaces, notably the Morse lemma asserting that quasigeodesics are close to
geodesics, the invariance of hyperbolicity under quasi-isometries, we
define and study the Gromov boundary and its associated distance, and prove
that a quasi-isometry between Gromov hyperbolic spaces extends to a
homeomorphism of the boundaries. We also classify the isometries of
hyperbolic spaces into elliptic, parabolic and loxodromic ones, both in
terms of translation length and of fixed points at infinity. We also prove
a less classical theorem, by Bonk and Schramm, asserting that a Gromov
hyperbolic space embeds isometrically in a geodesic Gromov-hyperbolic
space. As the original proof uses a transfinite sequence of Cauchy
completions, this is an interesting formalization exercise. Along the way,
we introduce basic material on isometries, quasi-isometries, geodesic
spaces, the Hausdorff distance, the Cauchy completion of a metric space,
and the exponential on extended real numbers.
\end{abstract}

\tableofcontents

% sane default for proof documents
\parindent 0pt\parskip 0.5ex

% generated text of all theories
\input{session}

% optional bibliography
\bibliographystyle{amsalpha}
\bibliography{root}

\end{document}

%%% Local Variables:
%%% mode: latex
%%% TeX-master: t
%%% End:
