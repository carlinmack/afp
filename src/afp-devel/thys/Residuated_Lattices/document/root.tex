\documentclass[11pt,a4paper]{article}
\usepackage{isabelle,isabellesym}
\usepackage{amssymb}
\usepackage[only,bigsqcap]{stmaryrd}

% this should be the last package used
\usepackage{pdfsetup}

% urls in roman style, theory text in math-similar italics
\urlstyle{rm}
\isabellestyle{it}


\begin{document}

\title{Residuated Lattices}
\author{Victor B. F. Gomes \and Georg Struth \\
  Department of Computer Science, University of Sheffield}
\maketitle

\begin{abstract}
  The theory of residuated lattices, first proposed by Ward and Dilworth~\cite{Ward39},
  is formalised in Isabelle/HOL.
  This includes concepts of residuated functions; their adjoints and conjugates.
  It also contains necessary and sufficient conditions for the existence of these operations
  in an arbitrary lattice.
  The mathematical components for residuated lattices are linked to the AFP entry for relation algebra.
  In particular, we prove J{\'o}nsson and Tsinakis~\cite{Jonsson93} conditions for a residuated
  boolean algebra to form a relation algebra.
\end{abstract}

\tableofcontents

% sane default for proof documents
\parindent 0pt\parskip 0.5ex

\section {Introduction}

text {*
  These theory files formalise algebraic residuated structures. They are briefly and sparsely
  commented. More information can be found in the books by Galatos and \emph{al.}~\cite{Galatos07},
  or the originals papers by Ward and Dilworth~\cite{Ward39}, 
  Jonsson and Tsinakis~\cite{Jonsson93}, and Maddux~\cite{Maddux96}.
  
  The mathematical components for residuated lattices are linked to the AFP entry for relation algebra.  
  Residuated lattices are also important in the context of Pratt's action algebras, which are currently
  formalised whitin the AFP entry for Kleene algebra.
  We are planning to link Kleene algebras and action algebras with this entry in the future.
  
  Isabelle/HOL default notation for lattices is used whenever possible. 
  Nevertheless, we use $\cdot$ as the multiplicative symbol instead of $*$,
  which is the one used in Isabelle libraries.

% generated text of all theories
\input{session}

% optional bibliography
\bibliographystyle{abbrv}
\bibliography{root}

\end{document}

%%% Local Variables:
%%% mode: latex
%%% TeX-master: t
%%% End:
