\documentclass[11pt,a4paper]{article}
\usepackage{isabelle,isabellesym}

% further packages required for unusual symbols (see also
% isabellesym.sty), use only when needed

\usepackage{amssymb}
  %for \<leadsto>, \<box>, \<diamond>, \<sqsupset>, \<mho>, \<Join>,
  %\<lhd>, \<lesssim>, \<greatersim>, \<lessapprox>, \<greaterapprox>,
  %\<triangleq>, \<yen>, \<lozenge>

%\usepackage{eurosym}
  %for \<euro>

%\usepackage[only,bigsqcap]{stmaryrd}
  %for \<Sqinter>

%\usepackage{eufrak}
  %for \<AA> ... \<ZZ>, \<aa> ... \<zz> (also included in amssymb)

%\usepackage{textcomp}
  %for \<onequarter>, \<onehalf>, \<threequarters>, \<degree>, \<cent>,
  %\<currency>

% this should be the last package used
\usepackage{pdfsetup}

% urls in roman style, theory text in math-similar italics
\urlstyle{rm}
\isabellestyle{it}

% for uniform font size
%\renewcommand{\isastyle}{\isastyleminor}


\begin{document}

\title{Quasi-Borel Spaces}
\author{Michikazu Hirata, Yasuhiko Minamide, Tetsuya Sato}
\maketitle

\begin{abstract}
  The notion of quasi-Borel spaces was introduced by Heunen et al.~\cite{Heunen_2017}.
  The theory provides a suitable denotational model
  for higher-order probabilistic programming languages with continuous distributions.

  This entry is a formalization of the theory of quasi-Borel spaces, including
  construction of quasi-Borel spaces (product, coproduct, function spaces),
  the adjunction between the category of measurable spaces and the category of quasi-Borel spaces,
  and the probability monad on quasi-Borel spaces.
  This entry also contains the formalization of the Bayesian regression presented in the work of Heunen et al.
  
  This work is a part of the work by same authors,
  \textit{Program Logic for Higher-Order Probabilistic Programs in Isabelle/HOL},
  which will be published in proceedings of the 16th International Symposium on Functional and Logic Programming (FLOPS 2022).
\end{abstract}

\tableofcontents

% sane default for proof documents
\parindent 0pt\parskip 0.5ex

% generated text of all theories
\input{session}

% optional bibliography
\bibliographystyle{abbrv}
\bibliography{root}

\end{document}

%%% Local Variables:
%%% mode: latex
%%% TeX-master: t
%%% End:
