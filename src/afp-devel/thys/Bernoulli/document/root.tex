\documentclass[11pt,a4paper]{article}
\usepackage[T1]{fontenc}
\usepackage{isabelle,isabellesym}

% further packages required for unusual symbols (see also
% isabellesym.sty), use only when needed

\usepackage{amssymb, amsmath}
  %for \<leadsto>, \<box>, \<diamond>, \<sqsupset>, \<mho>, \<Join>,
  %\<lhd>, \<lesssim>, \<greatersim>, \<lessapprox>, \<greaterapprox>,
  %\<triangleq>, \<yen>, \<lozenge>

%\usepackage{eurosym}
  %for \<euro>

%\usepackage[only,bigsqcap]{stmaryrd}
  %for \<Sqinter>

%\usepackage{eufrak}
  %for \<AA> ... \<ZZ>, \<aa> ... \<zz> (also included in amssymb)

%\usepackage{textcomp}
  %for \<onequarter>, \<onehalf>, \<threequarters>, \<degree>, \<cent>,
  %\<currency>

% this should be the last package used
\usepackage{pdfsetup}

% urls in roman style, theory text in math-similar italics
\urlstyle{rm}
\isabellestyle{it}

% for uniform font size
%\renewcommand{\isastyle}{\isastyleminor}


\begin{document}

\title{Bernoulli Numbers}
\author{Lukas Bulwahn and Manuel Eberl}
\maketitle

\begin{abstract}
Bernoulli numbers were first discovered in the closed-form expansion of 
the sum $1^m + 2^m + \ldots + n^m$ for a fixed $m$ and appear in many other places. This entry provides three different definitions for them: a recursive one, an explicit one, and one through their exponential generating function.

In addition, we prove some basic facts, e.\,g.\ their relation to sums of powers of integers and that all odd Bernoulli numbers except the first are zero. 
	We also prove the correctness of the Akiyama--Tanigawa algorithm~\cite{kaneko2000} for computing Bernoulli numbers with reasonable efficiency, and we define the periodic Bernoulli polynomials (which appear e.\,g.\ in the Euler--MacLaurin summation formula and the expansion of the log-Gamma function) and prove their basic properties.
\end{abstract}

\tableofcontents

\parindent 0pt\parskip 0.5ex

\input{session}

\bibliographystyle{abbrv}
\begingroup
\raggedright
\bibliography{root}
\endgroup

\end{document}

%%% Local Variables:
%%% mode: latex
%%% TeX-master: t
%%% End:
