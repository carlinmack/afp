\documentclass[11pt,a4paper]{article}
\usepackage[T1]{fontenc}
\usepackage{isabelle,isabellesym}
\usepackage{amssymb}
\usepackage[english]{babel}
\usepackage[only,bigsqcap]{stmaryrd}

% this should be the last package used
\usepackage{pdfsetup}

% urls in roman style, theory text in math-similar italics
\urlstyle{rm}
\isabellestyle{it}

% Tweaks
\newcounter{TTStweak_tag}
\setcounter{TTStweak_tag}{0}
\newcommand{\setTTS}{\setcounter{TTStweak_tag}{1}}
\newcommand{\resetTTS}{\setcounter{TTStweak_tag}{0}}
\newcommand{\insertTTS}{\ifnum\value{TTStweak_tag}=1 \ \ \ \fi}

\renewcommand{\isakeyword}[1]{\resetTTS\emph{\bf\def\isachardot{.}\def\isacharunderscore{\isacharunderscorekeyword}\def\isacharbraceleft{\{}\def\isacharbraceright{\}}#1}}
\renewcommand{\isachardoublequoteopen}{\insertTTS}
\renewcommand{\isachardoublequoteclose}{\setTTS}
\renewcommand{\isanewline}{\mbox{}\par\mbox{}\resetTTS}

\renewcommand{\isamarkupcmt}[1]{\hangindent5ex{\isastylecmt --- #1}}


\begin{document}

\title{The CAVA Automata Library}
\author{Peter Lammich}
\maketitle

\begin{abstract}
  We report on the graph and automata library that is used in the fully verified LTL model checker CAVA.
  As most components of CAVA use some type of graphs or automata,
  a common automata library simplifies assembly of the components and reduces redundancy.

  The CAVA Automata Library provides a hierarchy of graph and automata classes, together with some standard algorithms.
  Its object oriented design allows for sharing of algorithms, theorems, and implementations between its classes,
  and also simplifies extensions of the library.
  Moreover, it is integrated into the Automatic Refinement Framework, supporting automatic refinement of the abstract automata types
  to efficient data structures.

  Note that the CAVA Automata Library is work in progress. Currently, it is very specifically tailored towards the requirements of the CAVA model checker.
  Nevertheless, the formalization techniques presented here allow an extension of the library to a wider scope. Moreover, they are not limited to graph libraries,
  but apply to class hierarchies in general.

  The CAVA Automata Library is described in the paper: Peter Lammich, The CAVA Automata Library, Isabelle Workshop 2014, to appear.
\end{abstract}

\clearpage

\tableofcontents

\clearpage

% sane default for proof documents
\parindent 0pt\parskip 0.5ex

% generated text of all theories
\input{session}

\clearpage

% % optional bibliography
% \bibliographystyle{abbrv}
% \bibliography{root}

\end{document}

%%% Local Variables:
%%% mode: latex
%%% TeX-master: t
%%% End:
