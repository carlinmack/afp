\documentclass[11pt,a4paper]{article}
\usepackage[T1]{fontenc}
\usepackage{isabelle,isabellesym}

% this should be the last package used
\usepackage{pdfsetup}
\usepackage[english]{babel}

% urls in roman style, theory text in math-similar italics
\urlstyle{rm}
\isabellestyle{it}


\begin{document}

\title{Xml\thanks{This research is supported by FWF (Austrian Science Fund) projects J3202 and P22767.}}
\author{Christian Sternagel and Ren\'e Thiemann}
\maketitle

\begin{abstract}
This entry provides an ``XML library'' for Isabelle/HOL. This includes parsing
and pretty printing of XML trees as well as combinators for transforming XML
trees into arbitrary user-defined data. The main contribution of this entry is
an interface (fit for code generation) that allows for communication between
verified programs formalized in Isabelle/HOL and the outside world via XML. This
library was developed as part of the IsaFoR/CeTA project to which we refer for
examples of its usage. 
\end{abstract}

\tableofcontents

% sane default for proof documents
\parindent 0pt\parskip 0.5ex

% generated text of all theories
\input{session}

\end{document}

%%% Local Variables:
%%% mode: latex
%%% TeX-master: t
%%% End:
