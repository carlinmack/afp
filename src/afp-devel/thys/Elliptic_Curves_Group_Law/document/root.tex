\documentclass[11pt,a4paper]{article}
\usepackage{isabelle,isabellesym}

% further packages required for unusual symbols (see also
% isabellesym.sty), use only when needed

\usepackage[english]{babel}

% this should be the last package used
\usepackage{pdfsetup}

% urls in roman style, theory text in math-similar italics
\urlstyle{rm}
\isabellestyle{it}

% for uniform font size
%\renewcommand{\isastyle}{\isastyleminor}


\begin{document}

\title{The Group Law for Elliptic Curves}
\author{Stefan Berghofer}
\maketitle

\begin{abstract}
We prove the group law for elliptic curves in Weierstrass form over fields of
characteristic greater than 2. In addition to affine coordinates, we also
formalize projective coordinates, which allow for more efficient computations.
By specializing the abstract formalization to prime fields, we can apply
the curve operations to parameters used in standard security protocols.
\end{abstract}

\tableofcontents

% sane default for proof documents
\parindent 0pt\parskip 0.5ex

\section{Introduction}

Elliptic curves play an important role in cryptography, since they allow to achieve
a security level that is comparable to that of RSA, while requiring a smaller key
size and less computation time. The primitive operation on elliptic curves is \emph{point
addition}. To ensure the proper functioning of cryptographic algorithms based on
elliptic curves, such as Diffie-Hellman key exchange (ECDH) or digital signatures
(ECDSA), it is important that the points on the curve form a group with respect
to point addition.

Our formalization of elliptic curves is based on earlier work by Laurent Th{\'e}ry in
Coq \cite{Coq-Elliptic}. Like its Coq counterpart, the Isabelle formalization uses decision
procedures for rings and fields based on reflection, which are executed using Isabelle's
code generator for efficiency reasons. The decision procedure for rings is due to Gr{\'e}goire
and Mahboubi \cite{Mahboubi-Gregoire-TPHOLs2005} and was ported from Coq to Isabelle by
Bernhard Haeupler.

The formalization exists in two flavours: one based on axiomatic type classes, and
another one based on locales. While the axiomatic type class version is more concise,
the locale version is more suitable for working with concrete rings or fields like
prime fields.

% generated text of all theories
\input{session}

% optional bibliography
\bibliographystyle{abbrv}
\bibliography{root}

\end{document}

%%% Local Variables:
%%% mode: latex
%%% TeX-master: t
%%% End:
