\documentclass[11pt,a4paper]{article}

\usepackage[english]{babel}
\usepackage[utf8]{inputenc}

\usepackage{isabelle,isabellesym}
\usepackage{amssymb}
\usepackage[only,bigsqcap]{stmaryrd}

\usepackage[T1]{fontenc}

% this should be the last package used
\usepackage{pdfsetup}

% urls in roman style, theory text in math-similar italics
\urlstyle{rm}
\isabellestyle{it}

% for uniform font size
\renewcommand{\isastyle}{\isastyleminor}

\begin{document}

\title{Converting Linear Temporal Logic to Deterministic (Generalized) Rabin Automata}
\author{Salomon Sickert}
\maketitle

\begin{abstract}
Recently a new method directly translating linear temporal logic (LTL) formulas to deterministic (generalized) Rabin automata was described in \cite{DBLP:journals/fmsd/EsparzaKS16}. 

Compared to the existing approaches of constructing a non-deterministic Buechi-automaton in the first step and then applying a determinization procedure (e.g. some variant of Safra's construction) in a second step, this new approach preservers a relation between the formula and the states of the resulting automaton. While the old approach produced a monolithic structure, the new method is compositional. Furthermore it was shown in some cases the resulting automata were much smaller than the automata generated by existing approaches. In order to guarantee the correctness of the construction this entry contains a complete formalisation and verification of the translation. Furthermore from this basis executable code is generated.
\end{abstract}

\tableofcontents

% sane default for proof documents
\parindent 0pt\parskip 0.5ex

% generated text of all theories
\input{session}

\bibliographystyle{abbrv}
\bibliography{root}

\end{document}

%%% Local Variables:
%%% mode: latex
%%% TeX-master: t
%%% End:
