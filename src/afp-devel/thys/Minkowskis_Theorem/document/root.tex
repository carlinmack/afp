\documentclass[11pt,a4paper]{article}
\usepackage{isabelle,isabellesym}
\usepackage{amsfonts, amsmath, amssymb}

% this should be the last package used
\usepackage{pdfsetup}

% urls in roman style, theory text in math-similar italics
\urlstyle{rm}
\isabellestyle{it}


\begin{document}

\title{Minkowski's Theorem}
\author{Manuel Eberl}
\maketitle

\begin{abstract}
	Minkowski's theorem relates a subset of $\mathbb{R}^n$, the Lebesgue measure, and the integer lattice $\mathbb{Z}^n$: It states that any convex subset of $\mathbb{R}^n$ with volume greater than $2^n$ contains at least one lattice point from $\mathbb{Z}^n\setminus\{0\}$, i.\,e. a non-zero point with integer coefficients.

	A related theorem which directly implies this is Blichfeldt's theorem, which states that any subset of $\mathbb{R}^n$ with a volume greater than 1 contains two different points whose difference vector has integer components.

The entry contains a proof of both theorems.
\end{abstract}
\nocite{dummit}

\tableofcontents
\newpage
\parindent 0pt\parskip 0.5ex

\input{session}

\bibliographystyle{abbrv}
\bibliography{root}

\end{document}

%%% Local Variables:
%%% mode: latex
%%% TeX-master: t
%%% End:
