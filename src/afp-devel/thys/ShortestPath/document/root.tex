\documentclass[11pt,a4paper]{article}
\usepackage{isabelle,isabellesym}

% this should be the last package used
\usepackage{pdfsetup}
\usepackage{amssymb,amsmath,amsthm}
\newcommand{\real}{\mathbb{R}}

% urls in roman style, theory text in math-similar italics
\urlstyle{rm}
\isabellestyle{it}


\begin{document}

\title{An Axiomatic Characterization of the Single-Source Shortest Path Problem}
\author{By Christine Rizkallah}
\maketitle

\begin{abstract}
 % We provide an axiomatic characterization of the single-source shortest path problem. 
 This theory is split into two sections. In the first section, we give a formal proof that a well-known axiomatic characterization of the single-source shortest path problem is correct. Namely, we prove that in a directed graph $G=(V,E)$ with a non-negative cost function on the edges the single-source shortest path function $\mu:V\to\real\cup\{\infty\}$ is the only function that satisfies a set of four axioms. The first axiom states that the distance from the source vertex $s$ to itself should be equal to zero. The second states that the distance from $s$ to a vertex $v\in V$ should be infinity if and only if there is no path from $s$ to $v$. The third axiom is called triangle inequality and states that if there is a path from $s$ to $v$, and an edge $(u,v)\in E$, the distance from $s$ to $v$ is less than or equal to the distance from $s$ to $u$ plus the cost of $(u,v)$. The last axiom is called justification, it states that for every vertex $v$ other than $s$, if there is a path $p$ from $s$ to $v$ in $G$, then there is a predecessor edge $(u,v)$ on $p$ such that the distance from $s$ to $v$ is equal to the distance from $s$ to $u$ plus the cost of $(u,v)$. 

In the second section, we give a formal proof of the correctness of an axiomatic characterization of the single-source shortest path problem for directed graphs with general cost functions $c:E\to\real$. The axioms here are more involved because we have to account for potential negative cycles in the graph. The axioms are summarized in the three isabelle locales. 
\end{abstract}

\tableofcontents

% sane default for proof documents
\parindent 0pt\parskip 0.5ex

% generated text of all theories
\input{session}

\bibliographystyle{abbrv}
\bibliography{root}

\end{document}

%%% Local Variables:
%%% mode: latex
%%% TeX-master: t
%%% End:
