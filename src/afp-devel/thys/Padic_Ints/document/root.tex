\documentclass[11pt,a4paper]{article}
\usepackage{isabelle,isabellesym, amsmath, amssymb, amsfonts}

% further packages required for unusual symbols (see also
% isabellesym.sty), use only when needed

%\usepackage{amssymb}
  %for \<leadsto>, \<box>, \<diamond>, \<sqsupset>, \<mho>, \<Join>,
  %\<lhd>, \<lesssim>, \<greatersim>, \<lessapprox>, \<greaterapprox>,
  %\<triangleq>, \<yen>, \<lozenge>

%\usepackage{eurosym}
  %for \<euro>

%\usepackage[only,bigsqcap]{stmaryrd}
  %for \<Sqinter>

%\usepackage{eufrak}
  %for \<AA> ... \<ZZ>, \<aa> ... \<zz> (also included in amssymb)

%\usepackage{textcomp}
  %for \<onequarter>, \<onehalf>, \<threequarters>, \<degree>, \<cent>,
  %\<currency>

% this should be the last package used
\usepackage{pdfsetup}

% urls in roman style, theory text in math-similar italics
\urlstyle{rm}
\isabellestyle{it}

% for uniform font size
%\renewcommand{\isastyle}{\isastyleminor}


\begin{document}

\title{$p$-adic Hensel's Lemma}
\author{Aaron Crighton}
\maketitle

\tableofcontents

% sane default for proof documents
\parindent 0pt\parskip 0.5ex

\begin{abstract}
We formalize the ring of $p$-adic integers within the framework of the HOL-Algebra library. The carrier of the ring $\mathbb{Z}_p$ is formalized as the inverse limit of the residue rings $\mathbb{Z}/p^n\mathbb{Z}$ for a fixed prime $p$. We define a locale for reasoning about $\mathbb{Z}_p$ for a fixed prime $p$, and define an integer-valued valuation, as well as an extended-integer valued valuation on $\mathbb{Z}_p$ (where $0 \in \mathbb{Z}_p$ is the unique ring element mapped to $\infty$). Basic topological facts about the $p$-adic integers are formalized, including the completeness and sequential compactness of $\mathbb{Z}_p$. Taylor expansions of polynomials over a commutative ring are defined, culminating in the formalization of Hensel's Lemma based on a proof due to Keith Conrad \cite{keithconrad}. 
\end{abstract}
% generated text of all theories
\input{session}

% optional bibliography
\bibliographystyle{abbrv}
\bibliography{root}

\end{document}

%%% Local Variables:
%%% mode: latex
%%% TeX-master: t
%%% End:
