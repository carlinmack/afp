\documentclass[11pt,a4paper]{article}
\usepackage{isabelle,isabellesym}
\usepackage{amsfonts, amsmath, amssymb}

% this should be the last package used
\usepackage{pdfsetup}
\usepackage[shortcuts]{extdash}

% urls in roman style, theory text in math-similar italics
\urlstyle{rm}
\isabellestyle{rm}


\begin{document}

\title{Van der Waerden's Theorem}
\author{Katharina Kreuzer, Manuel Eberl}
\maketitle

\begin{abstract}
This article formalises the proof of Van der Waerden's Theorem from Ramsey theory. 

Van der Waerden's Theorem states that for integers $k$ and $l$ there exists a number $N$ which guarantees that if an integer interval of length at least $N$ is coloured with $k$ colours, there will always be an arithmetic progression of length $l$ of the same colour in said interval. 
The proof goes along the lines of Swan~\cite{Swan}.

The smallest number $N_{k,l}$ fulfilling Van der Waerden's Theorem is then called the Van der Waerden Number. Finding the Van der Waerden Number is still an open problem for most values of $k$ and~$l$.
\end{abstract}


\newpage
\tableofcontents

\newpage
\parindent 0pt\parskip 0.5ex

\input{session}

\nocite{Swan}

\bibliographystyle{abbrv}
\bibliography{root}

\end{document}

