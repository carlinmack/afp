\documentclass[11pt,a4paper]{article}
\usepackage{isabelle,isabellesym}
\usepackage{amsmath}
\usepackage{amssymb}

% this should be the last package used
\usepackage{pdfsetup}

% urls in roman style, theory text in math-similar italics
\urlstyle{rm}
\isabellestyle{it}


\begin{document}

\title{Evaluate Winding Numbers through Cauchy Indices}
\author{Wenda Li}
\maketitle

\begin{abstract}
	In complex analysis, the winding number measures the number of times a path (counterclockwise) winds around a point, while the Cauchy index can approximate how the path winds. This entry provides a formalisation of the Cauchy index, which is then shown to be related to the winding number. In addition, this entry also offers a tactic that enables users to evaluate the winding number by calculating Cauchy indices. The connection between the winding number and the Cauchy index can be found in the literature \cite{eisermann2012fundamental} \cite[Chapter 11]{rahman2002analytic}.
\end{abstract}

%\tableofcontents

% include generated text of all theories
\input{session}

\section{Acknowledgements}
The work was supported by the ERC Advanced Grant ALEXANDRIA (Project 742178), funded by the European Research Council
and led by Professor Lawrence Paulson at the University of Cambridge, UK.

\bibliographystyle{abbrv}
\bibliography{root}

\end{document}
