\documentclass[11pt,a4paper]{article}
\usepackage{isabelle,isabellesym}

% further packages required for unusual symbols (see also
% isabellesym.sty), use only when needed

%\usepackage{amssymb}
  %for \<leadsto>, \<box>, \<diamond>, \<sqsupset>, \<mho>, \<Join>,
  %\<lhd>, \<lesssim>, \<greatersim>, \<lessapprox>, \<greaterapprox>,
  %\<triangleq>, \<yen>, \<lozenge>

%\usepackage{eurosym}
  %for \<euro>

\usepackage[only,bigsqcap]{stmaryrd}
  %for \<Sqinter>

%\usepackage{eufrak}
  %for \<AA> ... \<ZZ>, \<aa> ... \<zz> (also included in amssymb)

%\usepackage{textcomp}
  %for \<onequarter>, \<onehalf>, \<threequarters>, \<degree>, \<cent>,
  %\<currency>

% this should be the last package used
\usepackage{pdfsetup}

% urls in roman style, theory text in math-similar italics
\urlstyle{rm}
\isabellestyle{it}

% for uniform font size
%\renewcommand{\isastyle}{\isastyleminor}


\begin{document}

\title{Properties of Orderings and Lattices}
\author{Georg Struth}

\maketitle

\begin{abstract}
  These components add further fundamental order and lattice-theoretic
  concepts and properties to Isabelle's libraries.  They follow by and
  large the introductory sections of the \emph{Compendium of
    Continuous Lattices},  covering directed and filtered sets,
  down-closed and up-closed sets, ideals and filters, Galois
  connections, closure and co-closure operators. Some emphasis is on
  duality and morphisms between structures---as in the Compendium.  To
  this end, three ad-hoc approaches to duality are compared.
\end{abstract}


\tableofcontents

\section{Introductory Remarks}

Basic order- and lattice-theoretic concepts are well covered in
Isabelle's libraries, and widely used. More advanced components are
spread out over various sites
(e.g.~\cite{Wenzel,Preoteasa11a,Preoteasa11b,ArmstrongS12,GomesS15,Ballarin}).

This formalisation takes the initial steps towards a modern structural
approach to orderings and lattices, as for instance in
denotational semantics of programs, algebraic logic or pointfree
topology.  Building on the components for orderings and lattices in
Isabelle's main libraries, it follows the classical textbook \emph{A
  Compendium of Continuous Lattices}~\cite{GierzHKLMS80} and, to a
lesser extent, Johnstone's monograph on \emph{Stone
  Spaces}~\cite{Johnstone82}. By integrating material from
other sources and extending it, a formalisation of undergraduate-level
textbook material on orderings and lattices might eventually emerge.

In the textbooks mentioned, concepts such as dualities, isomorphisms
between structures and relationships between categories are
emphasised.  These are essential to modern mathematics beyond
orderings and lattices; their formalisation with interactive theorem
provers is therefore of wider interest. Nevertheless such notions seem
rather underexplored with Isabelle, and I am not aware of a standard
way of modelling and using them. The present setting is perhaps the
simplest one in which their formalisation can be studied.

These components use Isabelle's axiomatic approach without carrier
sets. This is certainly a limitation, but it can be taken quite
far. Yet well known facts such as Tarski's theorem---the set of
fixpoints of an isotone endofunction on a complete lattice forms a
complete lattice---seem hard to formalise with it (at least without
using recent experimental extensions~\cite{Kuncar016}).

Firstly, leaner versions of complete lattices are introduced:
Sup-lattices (and their dual Inf-lattices), in which only Sups (or
Infs) are axiomatised, whereas the remaining operators, which are
axiomatised in the standard Isabelle class for complete lattices, are
defined explicitly.  This not only reduces of proof obligations in
instantiation or interpretation proofs, it also helps in constructions
where only suprema are represented faithfully (e.g. using morphisms
that preserve sups, but not infs, or vice versa).  At the moment,
Sup-lattices remain rather loosely integrated into Isabelle's lattice
hierarchy; a tighter one seems rather delicate.

Order and lattice duality is modelled, rather ad hoc, within a type
class that can be added to those for orderings and lattices.  Duality
thus becomes a functor that reverses the order and maps Sups to Infs
and vice versa, as expected. It also maps order-preserving functions
to order-preserving functions, Sup-preserving to Inf-preserving ones
and vice versa. This simple approach has not yet been optimised for
automatic generation of dual statements (which seems hard to achieve
anyway). It works quite well on simple examples.

The class-based approach to duality is contrasted by an implicit,
locale-based one (which is quite standard in Isabelle), and Wenzel's
data-type-based one~\cite{Wenzel}. Wenzel's approach generates many
properties of the duality functor automatically from Isabelle's data
type package. However, duality is not involutive, and this limits the
dualisation of theorems quite severely. The local-based approach
dualises theorems within the context of a type class or locale highly
automatically.  But, unlike the present approach, it is limited to
such contexts. Yet another approach to duality has been taken in
HOL-Algebra~\cite{Ballarin}, but it is essentially based on set theory
and therefore beyond the reach of simple axiomatic type classes.

The components presented also cover fundamental concepts such as
directed and filtered sets, down-closed and up-closed sets, ideals and
filters, notions of sup-closure and inf-closure, sup-preservation and
inf-preservation, properties of adjunctions (or Galois connections)
between orderings and (complete) lattices, fusion theorems for least
and greatest fixpoints, and basic properties of closure and co-closure
(kernel) operations, following the Compendium (most of these concepts
come as dual pairs!). As in this monograph, emphasis lies on
categorical aspects, but no formal category theory is used. In
addition, some simple representation theorems have been formalised,
including Stone's theorem for atomic boolean algebras (objects only).
The non-atomic case seems possible, but is left for future
work. Dealing with opposite maps properly, which is essential for
dualities, remains an issue.

Finally, in Isabelle's main libraries, complete distributive lattices
and complete boolean algebras are currently based on a very strong
distributivity law, which makes these structures \emph{completely
  distributive} and is basically an Axiom of Choice.  While powerset
algebras satisfy this law, other applications, for instance in
topology require different axiomatisations.  Complete boolean
algebras, in particular, are usually defined as complete lattices
which are also boolean algebras. Hence only a finite distributivity
law holds. Weaker distributivity laws are also essential for
axiomatising complete Heyting algebras (aka frames or locales), which
are relevant for point-free topology~\cite{Johnstone82}.

Many questions remain, in particular on tighter integrations
of duality and reasoning up to isomorphism with Isabelle and
beyond. In its present form, duality is often not picked up in the proofs
of more complex statements. Some statements from the Compendium and
Johnstone's book had to be ignored due to the absence of carrier sets
in Isabelle's standard components for orderings and lattices. Whether
Kuncar and Popescu's new types-to-sets translation~\cite{Kuncar016}
provides a satisfactory solution remains to be seen.

% sane default for proof documents
\parindent 0pt\parskip 0.5ex

% generated text of all theories
\input{session}

% optional bibliography
\bibliographystyle{abbrv}
\bibliography{root}

\end{document}

%%% Local Variables:
%%% mode: latex
%%% TeX-master: t
%%% End:
